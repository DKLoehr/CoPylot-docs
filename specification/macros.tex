
\usepackage[usenames,dvipsnames]{xcolor}

% Palette: defunct
% \definecolor{kw}{RGB}{138,43,226}
% \definecolor{func}{RGB}{233,105,0}
% \definecolor{microcode}{RGB}{143,188,187}

% Palette: Nord
% \definecolor{kw}{RGB}{94,129,172}
% \definecolor{func}{RGB}{191,97,106}
% \definecolor{microcode}{RGB}{180,142,173}

% Palette: Solarized
% \definecolor{kw}{HTML}{268BD2} % blue
% \definecolor{func}{HTML}{D33682} % magenta
% \definecolor{microcode}{HTML}{2AA198} % cyan
% \definecolor{microcode}{HTML}{B58900} % yellow
% \definecolor{microcode}{HTML}{D26937} % orange
% \definecolor{microcode}{HTML}{AF005F} % violet

% Palette: Gotham
\definecolor{kw}{HTML}{33859E} % cyan Q
\definecolor{func}{HTML}{C23127} % red Q
% \definecolor{microcode}{HTML}{D26937} % orange
% \definecolor{type}{HTML}{2AA889} % green
% \definecolor{microcode}{HTML}{4E5166} % violet
% \definecolor{magic}{HTML}{888CA6} % magenta
% \definecolor{microcode}{HTML}{EDB443} % yellow

% Palette: Slack
% \definecolor{kw}{HTML}{81D2E0} % blue
\definecolor{microcode}{HTML}{599F8C} % cyan Q
\definecolor{type}{HTML}{CC4876} % magenta Q
% \definecolor{magic}{HTML}{5C3A58} % violet Q
% \definecolor{type}{HTML}{EDB625} % yellow

\definecolor{flame}{RGB}{233,105,0}
\definecolor{gothamorange}{HTML}{D26937} % orange

% Palette: base-16-tomorrow
% @red:             #cc6666; // 08
% @orange:          #de935f; // 09
% @yellow:          #f0c674; // 0A
% @green:           #b5bd68; // 0B
% @cyan:            #8abeb7; // 0C
% @blue:            #81a2be; // 0D
% @purple:          #b294bb; // 0E
% @brown:           #a3685a; // 0F
% \definecolor{kw}{HTML}{81A2BE}
% \definecolor{func}{HTML}{CC6666}
% \definecolor{microcode}{HTML}{8ABEB7}
\definecolor{magic}{HTML}{B294BB}
% \definecolor{type}{HTML}{A3685A}

\usepackage{amsmath}
\usepackage{amssymb}
\usepackage{amsthm}
\usepackage{thmtools}
\usepackage{mathpartir}
\usepackage{stmaryrd}
\usepackage{xifthen}
\usepackage{relsize}
\usepackage{mathtools}
\usepackage{yfonts}
\usepackage{multicol}
\usepackage{hyperref}
\usepackage{catchfilebetweentags}
\usepackage[usenames,dvipsnames]{xcolor}

% Better TT font
\let\oldrmdefault\rmdefault
\let\oldsfdefault\sfdefault
\usepackage{lmodern}
\let\rmdefault\oldrmdefault
\let\sfdefault\oldsfdefault

%%% TeX convenience macros %%%%%%%%%%%%%%%%%%%%%%%%%%%%%%%%%%%%%%%%%%%%%%%%%%%%

% Produces the contents of the second argument only if the first argument is
% empty.
\newcommand{\ifnotempty}[2]{\ifthenelse{\isempty{#1}}{}{#2}}

\newcommand{\znote}[1]{\textit{\color{ForestGreen}(#1 -- ZP)}}
\newcommand{\tnote}[1]{\textit{\color{gothamorange}(#1 -- TC)}}

%%% General algebraic notation %%%%%%%%%%%%%%%%%%%%%%%%%%%%%%%%%%%%%%%%%%%%%%%%

\newcommand{\listConcat}{\ensuremath{\mathop{||}}}
\newcommand{\strConcat}{\ensuremath{\mathop{|}}}

%%% General logic notation %%%%%%%%%%%%%%%%%%%%%%%%%%%%%%%%%%%%%%%%%%%%%%%%%%%%

% Create a relational rule
% [#1] - Additional mathpartir arguments
% {#2} - Name of the rule
% {#3} - Premises for the rule
% {#4} - Conclusions for the rule
\newcommand{\relationRule}[4][]{\inferrule*[lab={\sc #2},#1]{#3}{#4}}

\declaretheorem[within=section]{lemma}
\declaretheorem[name=Definition,numberlike=lemma]{definition}
\declaretheorem[name=Notation,numberlike=lemma]{notation}
\declaretheorem[name=Theorem,numberlike=lemma]{theorem}

%%% Grammar definitions %%%%%%%%%%%%%%%%%%%%%%%%%%%%%%%%%%%%%%%%%%%%%%%%%%%%%%%

% Defines a grammar terminal.
% [#1] - The name of the command to define.  Defaults to "\gt#2".
% {#2} - The text of the grammar terminal.
\newcommand{\defgt}[2][]{%
    \ifthenelse{\isempty{#1}}{\defgt[\defgtname{#2}]{#2}}{%
        \expandafter\newcommand\expandafter{\csname #1\endcsname}{\defgtstyle{#2}}
    }%
}
\newcommand{\defgtstyle}[1]{\mathinner{\texttt{#1}}}
\newcommand{\defgtname}[1]{gt#1}

% Defines a grammar non-terminal.
% {#1} - The name of the command to define.
% {#2} - The math-mode text of the command.
\newcommand{\defgn}[2]{%
    \expandafter\newcommand\expandafter{\csname #1\endcsname}{\defgnstyle{#2}}%
}
\newcommand{\defgnstyle}[1]{\ensuremath{#1}}

% Defines an environment, {grammar}, for displaying language grammars.
% Within this environment, some commands are defined:
%   \grule[description]{non-terminal}{productions}
%   \gor (a symbol for separating productions)
%   \gline (a symbol for breaking a line of productions if more space is needed)
%   \setGrammarVertAdjustment{length} (which sets the vertical adjustment for
%                                      the next grammar newline)
\newcommand{\grammarNoteSpace}{\hspace{4mm}}
\newcounter{grammarnote}
\setcounter{grammarnote}{0}
\newcommand{\grammarDefs}{%
    \global\def\grammarVertAdjustment{0mm}%
    \newcommand{\gcomment}[1]{\hfill%
        \ifnum\value{grammarnote}=0
            \stepcounter{grammarnote}
            \grammarNoteSpace \textrm{\textsmaller{\itshape ##1}}
        \fi
    }%
    \newcommand{\grule}[3][]{%
        \setcounter{grammarnote}{0}
        ##2 & \ifnotempty{##3}{::=} & \newcommand{\gcommenttext}{##1} ##3 \hfill \gcomment{##1} \endgrule
    }%
    \newcommand{\gskip}{&&\endgrule}%
    \def\endgrule{\\[\grammarVertAdjustment]}%
    \newcommand{\gor}{\mathrel{\vert}}%
    \newcommand{\gline}{%
        \hfill \gcomment{\gcommenttext} \\[\grammarVertAdjustment] &&
    }%
}
\def\grammarColPad{\quad}%
\newenvironment{grammar}{%
    \begingroup%
    \grammarDefs%
    \begin{math}\begin{array}{@{}r@{\grammarColPad}c@{\grammarColPad}l@{}}%
}{%
    \end{array}\end{math}%
    \endgroup%
}

%%% CoPylot grammars %%%%%%%%%%%%%%%%%%%%%%%%%%%%%%%%%%%%%%%%%%%%%%%%%%%%%%%%%%

%%% Operational semantics

\defgn{olbl}{\ell}
\defgn{oglbl}{\overset{\star}{\ell}}
\defgn{ovalue}{v}
\defgn{ovariable}{x}
\defgn{odirective}{d}
\defgn{ostmt}{s}
\defgn{oprogram}{S}
\defgn{ostack}{T}
\defgn{ostackframe}{t}
\defgn{obinding}{B}
\defgn{oheap}{H}
\defgn{opscope}{\eta}
\defgn{oparent}{P}
\defgn{omem}{m}
\defgn{ogmem}{\overset{\star}{m}}
\defgn{oexpr}{e}
\defgn{oq}{\texttt{"}}
% Function related
\defgn{ofuncarr}{\rightarrow}
\defgn{omagicf}{\mathfrak{F}}
\defgn{omagicm}{\mathfrak{M}}
\defgn{ogenf}{F}
\defgn{ogenm}{M}
\defgn{osplat}{\ast}
% Code stack related
\defgn{omstack}{Y}
\defgn{omcode}{y}
\defgn{olstack}{Z}
\defgn{olcode}{z}

% These are wrong and should not emerge.
% \defgn{oenv}{E}
% \defgn{ofunc}{fun}
% \newcommand{\opscope}{{\color{red}\eta}}
\newcommand{\oenv}{{\color{red}E}}

\newlength\dunder
\settowidth{\dunder}{\_}

% Special
\newcommand{\osnone}{\texttt{None}}
\newcommand{\ostrue}{\text{\textsmaller{{\color{func}\sc True}}}}
\newcommand{\osfalse}{\text{\textsmaller{{\color{func}\sc False}}}}
% Relations
\newcommand{\osTransition}{\mathrel{\longrightarrow^1}}
\newcommand{\osBefore}[1]{\mathrel{\stackrel{{\scriptscriptstyle\boldsymbol{#1}}}{\lexicallyBefore}}}
% Helpers
\newcommand{\osLookup}{\text{\textsmaller{{\color{func}\sc Lookup}}}}
\newcommand{\osWrap}{\text{\textsmaller{{\color{func}\sc Wrap}}}}
\newcommand{\osGetCall}{\text{\textsmaller{{\color{func}\sc GetCall}}}}
% Format
\newcommand{\osLR}[1]{\langle #1 \rangle}
\newcommand{\osI}[2]{#1 &= #2}
\newcommand{\osInStack}[1]{\olstack \ifthenelse {\equal{#1}{}} {\listConcat} {\ifthenelse {\equal{#1}{\omstack'}}{\listConcat #1 \listConcat}{\listConcat [#1] \listConcat}} \omstack}
% Stylize
\newcommand{\okw}[1]{\;\text{\texttt{{\color{kw}#1}}}\;}
\newcommand{\osMC}[1]{\text{\textsmaller{{\color{microcode}\sc #1}}}}
\newcommand{\osFunc}[1]{\text{\textsmaller{{\color{func}\sc #1}}}}
\newcommand{\osType}[1]{\text{\textsmaller{{\color{type}\sc #1}}}}
\newcommand{\osMagic}[1]{\text{{\color{magic}\sc $\mathfrak{#1}$}}}
% -fix
\newcommand{\ostarvalue}[1]{\oq \star \ovariable_{\texttt{#1}} \oq}
\newcommand{\osInit}[1]{#1_{\text{\textsmaller{{\sc Init}}}}}
\newcommand{\osMem}[1]{\omem_{\texttt{#1}}}
\newcommand{\osClown}[1]{\rule{2\dunder}{0.4pt} #1 \rule{2\dunder}{0.4pt}}
% Magic
\newcommand{\osAssert}[3]{
  % #1: name, #2: value set, #3: type
  \relationRule{Assert #1}{%
    \omstack' =%
    \begin{cases}%
      [\ovalue], & \text{if}\; \ovalue #2 \cr%
      [\ovalue, \olcode_2, \ldots, \olcode_n, \osMC{Assert} \;n-1], & \text{otherwise}%
    \end{cases} %
  }{%
    \osInStack{\ovalue, \osMem{#3}, \olcode_2, \ldots, \olcode_n, \osMC{Assert} \;n}, \ostack, \oheap \osTransition%
    \osInStack{\omstack'}, \ostack, \oheap%
  }%
}

\newcommand{\gsAssert}[3]{
  % #1: name, #2: value set, #3: type
  \relationRule{Assert #1}{%
    \gnode_1 \gsBefore \gnode_2 \in \ggraph \\%
    \gnode_2 = \gsLR{\olbl, \osInStack{\ovalue, \osMem{#3}, \olcode_2, \ldots, \olcode_n, \osMC{Assert} \;n}, \gtime} \\%
    \omstack' =%
    \begin{cases}%
      [\ovalue], & \text{if}\; \ovalue #2 \cr%
      [\ovalue, \olcode_2, \ldots, \olcode_n, \osMC{Assert} \;n-1], & \text{otherwise}%
    \end{cases} %
  }{%
    \gsAddEdge{\gnode_2 \gsBefore \gsLR{\olbl, \osInStack{\omstack'}, \gtime + 1}}%
  }%
}

\newcommand{\osCast}[8]{
  % #1: magic name, #2: first type, #3: second type list
  % #4: len(list), #5: MI to call, #6: other things to put before the micro-stack,
  % #7: bind/assign #8: arg number
  \relationRule{#1}{%
    \ovalue = \osMagic{#1} \\%
    \ostack = [\osLR{\opscope, \olbl}] \listConcat \ostack' \\%
    \oprogram(\olbl) = \olbl:\oglbl': \ovariable \gteq \oexpr \\%
    #6%
    \omstack' = [%
    \ifthenelse {\equal{#7}{b}} {\opscope, } {}%
    %
    \ifthenelse {\equal{#8}{2}} {\omem_1, } {\ifthenelse {\equal{#8}{1}} {\omem, } {}}%
    %
    \osMC{Get}, \ostarvalue{value}, \osMC{Retrieve}, \osMC{Get}, #2, \\\\
    \osMC{Assert} \;1, %
    %
    \ifthenelse {\equal{#8}{2}} {\omem_2, \osMC{Get}, } {}%
    \ifthenelse {\equal{#8}{2}} {\ostarvalue{value}, \osMC{Retrieve}, \osMC{Get}, \ifthenelse {\equal{#3}{}} {} {#3, \osMC{Assert} \;#4, }} {}%
    %
    \osMC{#5}, \ovariable,%
    %
    \ifthenelse {\equal{#7}{b}} {\osMC{Bind}, } {\ifthenelse {\equal{#7}{a}} {\osMC{Assign}, } {}}%
    \osMC{Advance}]%
  }{%
    \olstack \listConcat [\ovalue, %
    %
    \ifthenelse {\equal{#8}{2}} {\omem_1, \omem_2, } {\ifthenelse {\equal{#8}{1}} {\omem, } {}}%
    %
    \opscope, \osMC{Call} \;#8], \listConcat \omstack, \ostack, \oheap \osTransition%
    \osInStack{\omstack'}, \ostack, \oheap%
  }%
}

\newcommand{\gsCast}[8]{
  % #1: magic name, #2: first type, #3: second type list
  % #4: len(list), #5: MI to call, #6: other things to put before the micro-stack,
  % #7: bind/assign #8: arg number
  \relationRule{#1}{%
    \gnode_1 \gsBefore \gnode_2 \in \ggraph \\%
    %
    \ifthenelse {\equal{#8}{2}} {%
      \gnode_2 = \gsLR{\olbl, \osInStack{\ovalue, \omem_1, \omem_2, \opscope, \osMC{Call} \;2}, \gtime}%
    } {%
      \ifthenelse {\equal{#8}{1}} {%
        \gnode_2 = \gsLR{\olbl, \osInStack{\ovalue, \omem, \opscope, \osMC{Call} \;1}, \gtime} %
      } {}%
    }%
    \\%
    \ovalue = \osMagic{#1} \\
    %
    \ifthenelse {\equal{#7}{b}} {\opscope = \gsFunc{Scopeof}(\gnode_2)} {}%
    \\
    \gsLR{\olbl', \omstack, \gtime'} = \gsFunc{Startof}(\gnode_2) \\
    \oprogram(\olbl') = \olbl':\oglbl'': \ovariable \gteq \oexpr \\
    #6%
    \omstack' = [%
    \ifthenelse {\equal{#7}{b}} {\opscope, } {}%
    %
    \ifthenelse {\equal{#8}{2}} {\omem_1, } {\ifthenelse {\equal{#8}{1}} {\omem, } {}}%
    %
    \osMC{Get}, \ostarvalue{value}, \osMC{Retrieve}, \osMC{Get}, #2, \\\\
    \osMC{Assert} \;1, %
    %
    \ifthenelse {\equal{#8}{2}} {\omem_2, \osMC{Get}, } {}%
    \ifthenelse {\equal{#8}{2}} {\ostarvalue{value}, \osMC{Retrieve}, \osMC{Get}, \ifthenelse {\equal{#3}{}} {} {#3, \osMC{Assert} \;#4, }} {}%
    %
    \osMC{#5}, \ovariable,%
    %
    \ifthenelse {\equal{#7}{b}} {\osMC{Bind}, } {\ifthenelse {\equal{#7}{a}} {\osMC{Assign}, } {}}%
    \osMC{Advance}]%
  }{
    \gsAddEdge{\gnode_2 \gsBefore \gsLR{\olbl, \osInStack{\omstack'}, \gtime + 1}}
  }

}

% Used in main
\newcommand{\osend}{\text{\textsmaller{{\color{func}\sc End}}}}
\newcommand{\osBind}{\text{\textsmaller{{\color{func}\sc Bind}}}}
\newcommand{\osMakeObj}{\text{\textsmaller{{\color{func}\sc MakeObj}}}}
\newcommand{\osCatch}{\text{\textsmaller{{\color{func}\sc Catch}}}}

%%% Graph semantics

\defgn{ggraph}{G}
\defgn{gedge}{g}
\defgn{gnode}{a}
\defgn{gtime}{\chi}
\defgn{genter}{\llparenthesis}
\defgn{gleave}{\rrparenthesis}
\defgn{gresult}{R}
\defgn{gvset}{V}

\newcommand{\niton}{\not\owns}

% Special
\newcommand{\gsstart}{\text{\textsmaller{{\color{func}\sc Start}}}}
\newcommand{\gsend}{\text{\textsmaller{{\color{func}\sc End}}}}
% Relations
\newcommand{\gsTransition}{\mathrel{\longrightarrow^1}}
\newcommand{\gsBefore}{\ll}
\newcommand{\gsSkip}{\lll}
\newcommand{\gsPrecede}{\ll \kern -4pt \ll}
% Format
\newcommand{\gsLR}[1]{\langle #1 \rangle}
\newcommand{\gsSet}[1]{\{ #1 \}}
%Stylize
\newcommand{\gsFunc}[1]{\text{\textsmaller{{\color{func}\sc #1}}}}

\newcommand{\gsAddEdge}[1]{\ggraph \gsTransition \ggraph \cup \gsSet{#1}}



%%% Abstract language grammar

\defgt[gtcolon]{:}
\defgt[gteq]{=}

\defgt[gtintplus]{int\ensuremath{{}^{\texttt{+}}}}
\defgt[gtintminus]{int\ensuremath{{}^{\texttt{-}}}}
\defgt[gtintzero]{int\ensuremath{{}^{\texttt{0}}}}

\newcommand{\xstmt}[3]{#1\gtcolon#2\gtcolon#3}

%%% Analysis grammar

\defgn{albl}{\hat{\ell}}
\defgn{avalue}{\hat{v}}
\defgn{avariable}{\hat{x}}
\defgn{adirective}{\hat{d}}
\defgn{astmt}{\hat{s}}
\defgn{aprogram}{\hat{S}}

% this will likely be replaced later when we move to a different kind of answer
\defgn{avalues}{\hat{V}}

\defgn{cfgnode}{\hat{o}}
\defgn{cfgedge}{\hat{g}}
\defgn{cfg}{\hat{G}}

\newcommand{\gstart}{\text{\textsmaller{\sc Start}}}
\newcommand{\gend}{\text{\textsmaller{\sc End}}}

\newcommand{\before}{\mathrel{\mathrlap{<}{\ <}}}
\newcommand{\lexicallyBefore}{\blacktriangleleft}

\defgn{astackelement}{\hat{k}}
\defgn{astack}{\hat{K}}

\newcommand{\kcapture}[1]{\text{\textsmaller{\textsc{Capture}}}(#1)}
\newcommand{\kjump}[1]{\text{\textsmaller{\textsc{Jump}}}(#1)}

%%% CoPylot functions and relations %%%%%%%%%%%%%%%%%%%%%%%%%%%%%%%%%%%%%%%%%%%

\newcommand{\isbefore}{\mathrel{\stackrel{{\scriptscriptstyle\boldsymbol{?}}}{\before}}}
\newcommand{\islexicallyBefore}{\mathrel{\stackrel{{\scriptscriptstyle\boldsymbol{?}}}{\lexicallyBefore}}}

\newcommand{\valueLookup}[3][\cfg]{#1(#2,#3)}

\newcommand{\cfgClosureStep}{\mathrel{\longrightarrow^1}}
\newcommand{\cfgClosureSteps}{\mathrel{\longrightarrow^*}}
