
\usepackage[usenames,dvipsnames]{xcolor}

% Palette: defunct
% \definecolor{kw}{RGB}{138,43,226}
% \definecolor{func}{RGB}{233,105,0}
% \definecolor{microcode}{RGB}{143,188,187}

% Palette: Nord
% \definecolor{kw}{RGB}{94,129,172}
% \definecolor{func}{RGB}{191,97,106}
% \definecolor{microcode}{RGB}{180,142,173}

% Palette: Solarized
% \definecolor{kw}{HTML}{268BD2} % blue
% \definecolor{func}{HTML}{D33682} % magenta
% \definecolor{microcode}{HTML}{2AA198} % cyan
% \definecolor{microcode}{HTML}{B58900} % yellow
% \definecolor{microcode}{HTML}{D26937} % orange
% \definecolor{microcode}{HTML}{AF005F} % violet

% Palette: Gotham
\definecolor{kw}{HTML}{33859E} % cyan Q
\definecolor{func}{HTML}{C23127} % red Q
% \definecolor{microcode}{HTML}{D26937} % orange
% \definecolor{type}{HTML}{2AA889} % green
% \definecolor{microcode}{HTML}{4E5166} % violet
% \definecolor{magic}{HTML}{888CA6} % magenta
% \definecolor{microcode}{HTML}{EDB443} % yellow

% Palette: Slack
% \definecolor{kw}{HTML}{81D2E0} % blue
\definecolor{microcode}{HTML}{599F8C} % cyan Q
\definecolor{type}{HTML}{CC4876} % magenta Q
% \definecolor{magic}{HTML}{5C3A58} % violet Q
% \definecolor{type}{HTML}{EDB625} % yellow

\definecolor{flame}{RGB}{233,105,0}
\definecolor{gothamorange}{HTML}{D26937} % orange

% Palette: base-16-tomorrow
% @red:             #cc6666; // 08
% @orange:          #de935f; // 09
% @yellow:          #f0c674; // 0A
% @green:           #b5bd68; // 0B
% @cyan:            #8abeb7; // 0C
% @blue:            #81a2be; // 0D
% @purple:          #b294bb; // 0E
% @brown:           #a3685a; // 0F
% \definecolor{kw}{HTML}{81A2BE}
% \definecolor{func}{HTML}{CC6666}
% \definecolor{microcode}{HTML}{8ABEB7}
\definecolor{magic}{HTML}{B294BB}
% \definecolor{type}{HTML}{A3685A}

\usepackage{amsmath}
\usepackage{amssymb}
\usepackage{amsthm}
\usepackage{thmtools}
\usepackage{mathpartir}
\usepackage{stmaryrd}
\usepackage{xifthen}
\usepackage{relsize}
\usepackage{mathtools}
\usepackage{yfonts}
\usepackage{multicol}
\usepackage{hyperref}
\usepackage{twoopt}
\usepackage{catchfilebetweentags}
\usepackage[usenames,dvipsnames]{xcolor}

% Better TT font
\let\oldrmdefault\rmdefault
\let\oldsfdefault\sfdefault
\usepackage{lmodern}
\let\rmdefault\oldrmdefault
\let\sfdefault\oldsfdefault

%%% TeX convenience macros %%%%%%%%%%%%%%%%%%%%%%%%%%%%%%%%%%%%%%%%%%%%%%%%%%%%

% Produces the contents of the second argument only if the first argument is
% empty.
\newcommand{\ifnotempty}[2]{\ifthenelse{\isempty{#1}}{}{#2}}

\newcommand{\znote}[1]{\textit{\color{ForestGreen}(#1 -- ZP)}}
\newcommand{\tnote}[1]{\textit{\color{gothamorange}(#1 -- TC)}}

%%% General algebraic notation %%%%%%%%%%%%%%%%%%%%%%%%%%%%%%%%%%%%%%%%%%%%%%%%

\newcommand{\listConcat}{\ensuremath{\mathop{||}}}
\newcommand{\strConcat}{\ensuremath{\mathop{|}}}

%%% General logic notation %%%%%%%%%%%%%%%%%%%%%%%%%%%%%%%%%%%%%%%%%%%%%%%%%%%%

% Create a relational rule
% [#1] - Additional mathpartir arguments
% {#2} - Name of the rule
% {#3} - Premises for the rule
% {#4} - Conclusions for the rule
\newcommand{\relationRule}[4][]{\inferrule*[lab={\sc #2},#1]{#3}{#4}}

\declaretheorem[within=section]{lemma}
\declaretheorem[name=Definition,numberlike=lemma]{definition}
\declaretheorem[name=Notation,numberlike=lemma]{notation}
\declaretheorem[name=Theorem,numberlike=lemma]{theorem}

%%% Grammar definitions %%%%%%%%%%%%%%%%%%%%%%%%%%%%%%%%%%%%%%%%%%%%%%%%%%%%%%%

% Defines a grammar terminal.
% [#1] - The name of the command to define.  Defaults to "\gt#2".
% {#2} - The text of the grammar terminal.
\newcommand{\defgt}[2][]{%
    \ifthenelse{\isempty{#1}}{\defgt[\defgtname{#2}]{#2}}{%
        \expandafter\newcommand\expandafter{\csname #1\endcsname}{\defgtstyle{#2}}
    }%
}
\newcommand{\defgtstyle}[1]{\mathinner{\texttt{\,#1\,}}}
\newcommand{\defgtname}[1]{gt#1}

% Defines a grammar keyword terminal.
% [#1] - The name of the command to define.  Defaults to "\gt#2".
% {#2} - The text of the grammar terminal.
\newcommand{\defkw}[2][]{%
    \ifthenelse{\isempty{#1}}{\defkw[\defkwname{#2}]{#2}}{%
        \expandafter\newcommand\expandafter{\csname #1\endcsname}{\defkwstyle{#2}}
    }%
}
\newcommand{\defkwstyle}[1]{\mathinner{\texttt{\color{kw}\,#1\,}}}
\newcommand{\defkwname}[1]{gt#1}

% Defines a grammar non-terminal.
% {#1} - The name of the command to define.
% {#2} - The math-mode text of the command.
\newcommand{\defgn}[2]{%
    \expandafter\newcommand\expandafter{\csname #1\endcsname}{\defgnstyle{#2}}%
}
\newcommand{\defgnstyle}[1]{\ensuremath{#1}}

% Defines an environment, {grammar}, for displaying language grammars.
% Within this environment, some commands are defined:
%   \grule[description]{non-terminal}{productions}
%   \gor (a symbol for separating productions)
%   \gline (a symbol for breaking a line of productions if more space is needed)
%   \setGrammarVertAdjustment{length} (which sets the vertical adjustment for
%                                      the next grammar newline)
\newcommand{\grammarNoteSpace}{\hspace{4mm}}
\newcounter{grammarnote}
\setcounter{grammarnote}{0}
\newcommand{\grammarDefs}{%
    \global\def\grammarVertAdjustment{0mm}%
    \newcommand{\gcomment}[1]{\hfill%
        \ifnum\value{grammarnote}=0
            \stepcounter{grammarnote}
            \grammarNoteSpace \textrm{\textsmaller{\itshape ##1}}
        \fi
    }%
    \newcommand{\grule}[3][]{%
        \setcounter{grammarnote}{0}
        ##2 & \ifnotempty{##3}{::=} & \newcommand{\gcommenttext}{##1} ##3 \hfill \gcomment{##1} \endgrule
    }%
    \newcommand{\gskip}{&&\endgrule}%
    \def\endgrule{\\[\grammarVertAdjustment]}%
    \newcommand{\gor}{\mathrel{\vert}}%
    \newcommand{\gline}{%
        \hfill \gcomment{\gcommenttext} \\[\grammarVertAdjustment] &&
    }%
}
\def\grammarColPad{\quad}%
\newenvironment{grammar}{%
    \begingroup%
    \grammarDefs%
    \begin{math}\begin{array}{@{}r@{\grammarColPad}c@{\grammarColPad}l@{}}%
}{%
    \end{array}\end{math}%
    \endgroup%
}

%%% CoPylot grammars %%%%%%%%%%%%%%%%%%%%%%%%%%%%%%%%%%%%%%%%%%%%%%%%%%%%%%%%%%

%%% Lamia Language Grammar

\defgn{olbl}{\ell}
\defgn{ovalvariable}{x}
\defgn{omemvariable}{y}
\defgn{ostmts}{S}
\defgn{ostmt}{s}
\defgn{odirective}{d}
\defgn{oexpr}{e}
\defgn{obinop}{\oplus}
\defgn{ounop}{\otimes}
\defgn{oheap}{H}
\defgn{otstmts}{T}
\defgn{otstmt}{t}

\defgt[gtcolon]{:}
\defgt{let}
\defgt[gteq]{=}
\defgt{alloc}
\defgt[gtopar]{(}
\defgt[gtcpar]{)}
\defgt[gtobrk]{[}
\defgt[gtcbrk]{]}
\defgt[gtobrc]{\char123} % {
\defgt[gtcbrc]{\char125} % }
\defgt[gtsemi]{;}
\defgt[gtarrow]{->}
\defgt[gtcomma]{,}
\defgt{fun}

\defkw{def}
\defkw{store}
\defkw{get}
\defkw{is}
\defkw{return}
\defkw{ifresult}
\defkw{raise}
\defkw{try}
\defkw{except}
\defkw{while}
\defkw{do}
\defkw{if}
\defkw{then}
\defkw{else}
\defkw{methodbind}

\defkw{not}
\defkw{isfunc}
\defkw{isint}
\defkw{isbool}

\defkw{haskey}
\defkw[gtintplus]{int+}
\defkw[gtintminus]{int-}
\defkw{and}
\defkw{or}
\defkw[gtlstconcat]{lst+}
\defkw[gtcmp]{==}

%%% Lamia Evaluation Grammar

\defgn{omem}{m}
\defgn{opscope}{\eta}
\defgn{ovalue}{v}
\defgn{obinding}{B}
\defgn{ogenf}{F}
\defgn{ogenm}{M}
\defgn{osplat}{\ast}

\newlength\dunder
\settowidth{\dunder}{\_}

% Special
\newcommand{\osnone}{\texttt{None}}
\newcommand{\ostrue}{\text{\textsmaller{{\color{func}\sc True}}}}
\newcommand{\osfalse}{\text{\textsmaller{{\color{func}\sc False}}}}
% Relations
\newcommand{\osTransition}{\mathrel{\longrightarrow^1}}
\newcommand{\osBefore}{\mathrel{\stackrel{{\scriptscriptstyle\boldsymbol{\ostmts}}}{\blacktriangleleft}}}
\newcommand{\osUnder}{\mathrel{\stackrel{{\scriptscriptstyle\boldsymbol{\ostmts}}}{\blacktriangledown}}}
\newcommand{\osStartof}{\mathrel{\stackrel{{\scriptscriptstyle\boldsymbol{\ostmts}}}{\blacktriangleleft \kern -4pt \blacktriangleleft}}}
% Format
\newcommand{\osLR}[1]{\langle #1 \rangle}
% Stylize
\newcommand{\osMC}[1]{\text{\textsmaller{{\color{microcode}\sc #1}}}}
\newcommand{\osFunc}[1]{\text{\textsmaller{{\color{func}\sc #1}}}}
\newcommand{\osType}[1]{\text{\textsmaller{{\color{type}\sc #1}}}}
\newcommand{\osMagic}[1]{\text{{\color{magic}\sc $\mathfrak{#1}$}}}
% -fix
\newcommand{\ostarvalue}[1]{\oq \star \ovariable_{\texttt{#1}} \oq}
\newcommand{\osClown}[1]{\rule{2\dunder}{0.4pt} #1 \rule{2\dunder}{0.4pt}}
\newcommand{\osBlock}[2][]{{\otstmts #1}_{\texttt{#2}}}

%%% Lamia Graph Grammar

\defgn{ggraph}{G}
\defgn{gedge}{g}
\defgn{gnode}{o}
\defgn{gstate}{a}
\defgn{gtime}{\chi}
\defgn{genter}{\llparenthesis}
\defgn{gleave}{\rrparenthesis}
\defgn{gresult}{R}
\defgn{gvset}{V}
\defgn{gstmt}{\overset{\star}{s}}

% Lookup

\defgn{lstack}{K}
\defgn{lstackelt}{k}
\defgn{lresult}{r}
\defgn{lresults}{R}
\defgn{linstr}{I}

% Time

\defgn{tplus}{\triangleleft}
\defgn{tpush}{\triangledown}
\defgn{tpop}{\triangle}
\defgn{treset}{\circledcirc}

\newcommand{\lInstr}[1]{\text{\textsmaller{{\color{linstr}\sc #1}}}}

% Special
\newcommand{\gsstart}{\text{\textsmaller{{\color{node}\sc Start}}}}
\newcommand{\gsend}{\text{\textsmaller{{\color{node}\sc End}}}}
\newcommandtwoopt{\gsadvance}[2][][]{\text{\textsmaller{{\color{node}\sc Advance }}} \osLR{#1, \gtime#2}}
\newcommandtwoopt{\gsreturn}[2][][]{\text{\textsmaller{{\color{node}\sc Return }}} \osLR{#1, \gtime#2}}
\newcommandtwoopt{\gsraise}[2][][]{\text{\textsmaller{{\color{node}\sc Raise }}} \osLR{#1, \gtime#2}}
\newcommandtwoopt{\gsifresult}[2][][]{\text{\textsmaller{{\color{node}\sc Ifresult }}} \osLR{#1, \gtime#2}}
% Relations
\newcommand{\gsTransition}{\mathrel{\longrightarrow^1}}
\newcommand{\gsBefore}{\rightarrowtriangle}
\newcommand{\gsSkip}{\lll}
\newcommand{\gsPrecede}{\ll \kern -4pt \ll}
% Format
\newcommand{\gsLR}[1]{\langle #1 \rangle}
\newcommand{\gsSet}[1]{\{ #1 \}}
% Stylize
\newcommand{\gsFunc}[1]{\text{\textsmaller{{\color{func}\sc #1}}}}
% Predicates
\newcommand{\gsBeforeIn}[1][]{%
  \ifthenelse{\equal{#1}{}}{%
    \overset{?}{\gsBefore}%
  }{%
    \overset{?}{\underset{#1}{\gsBefore}}%
  }%
}
% [\ifthenelse{\equal{#1}{}}{\ggraph}{#1}]

\newcommand{\gsAddEdge}[1]{\ggraph \gsTransition \ggraph \cup \gsSet{#1}}

\newcommand{\rulename}[1]{%
    \begingroup%
        \setlength{\fboxsep}{1.5pt}%
        \fcolorbox{black}{gray!15!white}{%
            \textsc{\textsmaller{#1}}%
        }%
        \vspace*{1pt}%
    \endgroup%
    \\%
}
\newcommand{\formalRuleLine}{\\ \hspace*{1.5em}}
