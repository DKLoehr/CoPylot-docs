%<*micro>
\gor    \osMC{Assert} \;n
\gor    \osMC{Sum}
\gor    \osMC{Neg}
\gor    \osMC{StrConcat}
\gor    \osMC{StrContains}
\gor    \osMC{Getitem}
% \gor    \osMC{Contains}
%</micro>

%<*grimoire>
          \osMagic{Call}
  \gor    \osMagic{Type}
  \gor    \osMagic{Bool}
  \gor    \osMagic{Getattribute}
  \gor    \osMagic{IntAdd}
  \gor    \osMagic{IntNeg}
  \gor    \osMagic{StrAdd}
  \gline
  \gor    \osMagic{StrContains}
  % \gor    \osMagic{LstIter}
  % \gor    \osMagic{LstContains}
  \gor    \osMagic{LstGetitem}
  % \gor    \osMagic{TplIter}
  % \gor    \osMagic{TplContains}
  % \gline
  \gor    \osMagic{TplGetitem}
%</grimoire>

%<*assertions>
\begin{mathpar}
  \osAssert {Method} {\in \ogenm} {MetType}
\end{mathpar}

\begin{mathpar}
  \osAssert {Int} {\in \mathbb{Z}} {IntType}
\end{mathpar}

\begin{mathpar}
  \osAssert {String} {\in \mathbb{S}} {StrType}
\end{mathpar}

\begin{mathpar}
  \osAssert {List} {\text{ is of form } [\omem, \ldots]} {LstType}
\end{mathpar}

\begin{mathpar}
  \osAssert {Tuple} {\text{ is of form } (\omem, \ldots)} {TplType}
\end{mathpar}

\begin{mathpar}
  \relationRule{Assert (base case)}{
    \\
  }{
    \osInStack{\osMC{Assert} \;0}, \ostack, \oheap \osTransition
    [\osMC{AllocTypeError}, \osMC{Raise}], \ostack, \oheap
  }
\end{mathpar}
%</assertions>

%<*unop>
\begin{mathpar}
  \relationRule{Bool \ovalue}{
    \ovalue' = \osFunc{Bool}(\ovalue)
  }{
    \osInStack{\ovalue, \osMC{Bool}}, \ostack, \oheap \osTransition
    \osInStack{\ovalue'}, \ostack, \oheap
  }
\end{mathpar}

\tnote{All those related to Bool should be handled in simplification, not here.}
%</unop>

%<*binop>
\begin{mathpar}
  \relationRule{Sum \ovalue \; \ovalue}{
    \ovalue = \ovalue_1 + \ovalue_2
  }{
    \osInStack{\ovalue_1, \ovalue_2, \osMC{Sum}}, \ostack, \oheap \osTransition
    \osInStack{\ovalue}, \ostack, \oheap
  }
\end{mathpar}

\begin{mathpar}
  \relationRule{Neg \ovalue}{
    \ovalue' = -(\ovalue)
  }{
    \osInStack{\ovalue, \osMC{Neg}}, \ostack, \oheap \osTransition
    \osInStack{\ovalue'}, \ostack, \oheap
  }
\end{mathpar}

\begin{mathpar}
  \relationRule{StrConcat \ovalue \; \ovalue}{
    \ovalue = \ovalue_1 \strConcat \ovalue_2
  }{
    \osInStack{\ovalue_1, \ovalue_2, \osMC{StrConcat}}, \ostack, \oheap \osTransition
    \osInStack{\ovalue}, \ostack, \oheap
  }
\end{mathpar}

\begin{mathpar}
  \relationRule{StrContains \ovalue}{
    \ovalue =
    \begin{cases}
      \ostrue, & \text{if } \ovalue_2 \text{ is a substring of } \ovalue_1 \cr
      \osfalse, & \text{otherwise}
    \end{cases}
  }{
    \osInStack{\ovalue_1, \ovalue_2, \osMC{StrContains}}, \ostack, \oheap \osTransition
    \osInStack{\ovalue}, \ostack, \oheap
  }
\end{mathpar}

\tnote{Iters do not exist yet.}

% \begin{mathpar}
%   \relationRule{Contains \ovalue}{
%     \ovalue =
%     \begin{cases}
%       \ostrue, & \text{if } \ovalue_2 \in \ovalue_1 \cr
%       \osfalse, & \text{otherwise}
%     \end{cases}
%   }{
%     \osInStack{\ovalue_1, \ovalue_2, \osMC{Contains}}, \ostack, \oheap \osTransition
%     \osInStack{\ovalue}, \ostack, \oheap
%   }
% \end{mathpar}

\begin{mathpar}
  \relationRule{Getitem \ovalue}{
    \ovalue_1 = [\omem_0, \ldots, \omem_n] \;|\; (\omem_0, \ldots, \omem_n) \\
    0 \le \ovalue_2 \le n \\
    \omstack' = [\omem_{\ovalue_2}, \osMC{Get}]
  }{
    \osInStack{\ovalue_1, \ovalue_2, \osMC{Getitem}}, \ostack, \oheap \osTransition
    \osInStack{\omstack'}, \ostack, \oheap
  }
\end{mathpar}

\begin{mathpar}
  \relationRule{Getitem (out of bound)}{
    \ovalue_1 = [\omem_0, \ldots, \omem_n] | (\omem_0, \ldots, \omem_n) \\
    0 > \ovalue_2 \;|\; \ovalue > n
  }{
    \osInStack{\ovalue_1, \ovalue_2, \osMC{Getitem}}, \ostack, \oheap \osTransition
    [\osMC{AllocIndexError}, \osMC{Raise}], \ostack, \oheap
  }
\end{mathpar}
%</binop>

%<*op>
\begin{mathpar}
  \relationRule{Wrong Number of Args}{
    \ovalue = \omagicm \;|\; \omagicf \\
    \osFunc{CheckArgs}(\ovalue, n) = \osfalse \\
    \omstack' = [\osMC{AllocTypeError}, \osMC{Raise}]
  }{
    \osInStack{\ovalue, \omem_1, \ldots, \omem_n, \opscope, \osMC{Call} \;n}, \ostack, \oheap \osTransition%
    \osInStack{\omstack'}, \ostack, \oheap
  }
\end{mathpar}

\begin{mathpar}
  \relationRule{Call}{
    \ovalue = \osMagic{Call} \\
    \omstack' = [\omem, \osMC{Get}, \ostarvalue{value}, \osMC{Retrieve}, \omem_1, \ldots, \omem_n, \osMC{Alloc}, \osMC{Convert} \;n, \osMC{Advance}]
    % For now, it calls its object's value field.
  }{
    \osInStack{\ovalue, \omem, \omem_1, \ldots, \omem_n, \opscope, \osMC{Call} \;n+1}, \ostack, \oheap \osTransition
    \osInStack{\omstack'}, \ostack, \oheap
  }
\end{mathpar}

\tnote{magic call to be updated}

\begin{mathpar}
  \relationRule{Type}{
    \ovalue = \osMagic{Type} \\
    \ostack = [\osLR{\opscope, \olbl}] \listConcat \ostack' \\
    \oprogram(\olbl) = \olbl:\oglbl': \ovariable \gteq \oexpr \\
    \omstack' = [\opscope, \omem, \osMC{Get}, \oq \osClown{class} \oq, \osMC{Retrieve}, \ovariable, \osMC{Bind}, \osMC{Advance}]
  }{
    \osInStack{\ovalue, \omem, \opscope, \osMC{Call} \;1}, \ostack, \oheap \osTransition
    \osInStack{\omstack'}, \ostack, \oheap
  }
\end{mathpar}

\begin{mathpar}
  \relationRule{Bool}{
    \ovalue = \osMagic{Bool} \\
    \omstack' = [\omem, \osMC{Get}, \ostarvalue{value}, \osMC{Retrieve}, \osMC{Get}, \osMC{Bool}, \osMC{Advance}]
  }{
    \osInStack{\ovalue, \omem, \opscope, \osMC{Call} \;1}, \ostack, \oheap \osTransition
    \osInStack{\omstack'}, \ostack, \oheap
  }
\end{mathpar}

\begin{mathpar}
  \osCast {IntAdd} {\osMem{IntType}} {\osMem{IntType}, \osMem{FloatType}} {2} {\osMC{Sum}} {} {a} {2}
\end{mathpar}

\begin{mathpar}
  \osCast {IntNeg} {\osMem{IntType}} {} {} {\osMC{Neg}} {} {a} {1}
\end{mathpar}

\begin{mathpar}
  \osCast {StrAdd} {\osMem{StrType}} {\osMem{StrType}} {1} {\osMC{StrConcat}} {} {a} {2}
\end{mathpar}

\begin{mathpar}
  \osCast {StrContains} {\osMem{StrType}} {\osMem{StrType}} {1} {\osMC{StrContains}} {} {a} {2}
\end{mathpar}

% \begin{mathpar}
%   \osDuoMagic {LstContains} {\osMem{LstType}} {} {} {\osMC{Contains}} {}
% \end{mathpar}

\begin{mathpar}
  \osCast {LstGetitem} {\osMem{LstType}} {\osMem{IntType}} {1} {\osMC{Getitem}} {} {b} {2}
\end{mathpar}

% \begin{mathpar}
%   \osDuoMagic {TplContains} {\osMem{TplType}} {} {} {\osMC{Contains}} {}
% \end{mathpar}
%</op>

%<*helper>
\begin{definition}[Magic argument check]
  \begin{flalign*}
    \osFunc{CheckArgs}(\ovalue, n) =
    \begin{cases}
      \ostrue, & \text{if } (\ovalue, n) =
        \begin{cases}
          (\osMagic{Type}, 1) \\
          (\osMagic{IntAdd}, 2) \\
          (\osMagic{IntNeg}, 1) \\
          (\osMagic{StrAdd}, 2) \\
          (\osMagic{StrContains}, 2) \\
          (\osMagic{LstGetitem}, 2) \\
          (\osMagic{TplGetitem}, 2) \\
        \end{cases}
      \\
      \osfalse, & \text{otherwise}
    \end{cases}
  \end{flalign*}
\end{definition}

\begin{definition}[Bool Evaluation]
  \begin{flalign*}
    \osFunc{Bool}(\ovalue) =
    \begin{cases}
      \osfalse, & \text{if } \ovalue \in \{0,\texttt{"\;"},\osfalse,{[\;]},(\;),\osnone\} \\
      \ostrue, & \text{otherwise}
    \end{cases}
  \end{flalign*}
\end{definition}
%</helper>

%<*graph-assertions>
\begin{mathpar}
  \gsAssert {Method} {\in \ogenm} {MetType}
\end{mathpar}

\begin{mathpar}
  \gsAssert {Int} {\in \mathbb{Z}} {IntType}
\end{mathpar}

\begin{mathpar}
  \gsAssert {String} {\in \mathbb{S}} {StrType}
\end{mathpar}

\begin{mathpar}
  \gsAssert {List} {\text{ is of form } [\omem, \ldots]} {LstType}
\end{mathpar}

\begin{mathpar}
  \gsAssert {Tuple} {\text{ is of form } (\omem, \ldots)} {TplType}
\end{mathpar}

\begin{mathpar}
  \relationRule{Assert (base case)}{
    \gnode_1 \gsBefore \gnode_2 \in \ggraph \\
    \gnode_2 = \gsLR{\olbl, \osInStack{\osMC{Assert} \;0}, \gtime} \\
    \omstack' = [\osMC{AllocTypeError}, \osMC{Raise}]
  }{
    \gsAddEdge{\gnode_2 \gsBefore \gsLR{\olbl, \osInStack{\omstack'}, \gtime + 1}}
  }
\end{mathpar}
%</graph-assertions>

%<*graph-unop>
\begin{mathpar}
  \relationRule{Bool \ovalue}{
    \gnode_1 \gsBefore \gnode_2 \in \ggraph \\
    \gnode_2 = \gsLR{\olbl, \osInStack{\ovalue, \osMC{Bool}}, \gtime} \\
    \ovalue' = \osFunc{Bool}(\ovalue)
  }{
    \gsAddEdge{\gnode_2 \gsBefore \gsLR{\olbl, \osInStack{\ovalue'}, \gtime + 1}}
  }
\end{mathpar}
%</graph-unop>

%<*graph-binop>
\begin{mathpar}
  \relationRule{Sum \ovalue \; \ovalue}{
    \gnode_1 \gsBefore \gnode_2 \in \ggraph \\
    \gnode_2 = \gsLR{\olbl, \osInStack{\ovalue_1, \ovalue_2, \osMC{Sum}}, \gtime} \\
    \ovalue = \ovalue_1 + \ovalue_2
  }{
    \gsAddEdge{\gnode_2 \gsBefore \gsLR{\olbl, \osInStack{\ovalue}, \gtime + 1}}
  }
\end{mathpar}

\begin{mathpar}
  \relationRule{Neg \ovalue}{
    \gnode_1 \gsBefore \gnode_2 \in \ggraph \\
    \gnode_2 = \gsLR{\olbl, \osInStack{\ovalue, \osMC{Neg}}, \gtime} \\
    \ovalue' = -(\ovalue)
  }{
    \osInStack{\ovalue, \osMC{Neg}}, \ostack, \oheap \osTransition
    \osInStack{\ovalue'}, \ostack, \oheap
  }
\end{mathpar}

\begin{mathpar}
  \relationRule{StrConcat \ovalue \; \ovalue}{
    \gnode_1 \gsBefore \gnode_2 \in \ggraph \\
    \gnode_2 = \gsLR{\olbl, \osInStack{\ovalue_1, \ovalue_2, \osMC{StrConcat}}, \gtime} \\
    \ovalue = \ovalue_1 \strConcat \ovalue_2
  }{
    \gsAddEdge{\gnode_2 \gsBefore \gsLR{\olbl, \osInStack{\ovalue}, \gtime + 1}}
  }
\end{mathpar}

\begin{mathpar}
  \relationRule{StrContains \ovalue}{
    \gnode_1 \gsBefore \gnode_2 \in \ggraph \\
    \gnode_2 = \gsLR{\olbl, \osInStack{\ovalue_1, \ovalue_2, \osMC{StrContains}}, \gtime} \\
    \ovalue =
    \begin{cases}
      \ostrue, & \text{if } \ovalue_2 \text{ is a substring of } \ovalue_1 \cr
      \osfalse, & \text{otherwise}
    \end{cases}
  }{
    \gsAddEdge{\gnode_2 \gsBefore \gsLR{\olbl, \osInStack{\ovalue}, \gtime + 1}}
  }
\end{mathpar}

\begin{mathpar}
  \relationRule{Getitem \ovalue}{
    \gnode_1 \gsBefore \gnode_2 \in \ggraph \\
    \gnode_2 = \gsLR{\olbl, \osInStack{\ovalue_1, \ovalue_2, \osMC{Getitem}}, \gtime} \\
    \ovalue_1 = [\omem_0, \ldots, \omem_n] \;|\; (\omem_0, \ldots, \omem_n) \\
    0 \le \ovalue_2 \le n \\
    \omstack' = [\omem_{\ovalue_2}, \osMC{Get}]
  }{
    \gsAddEdge{\gnode_2 \gsBefore \gsLR{\olbl, \osInStack{\omstack'}, \gtime + 1}}
  }
\end{mathpar}

\begin{mathpar}
  \relationRule{Getitem (out of bound)}{
    \gnode_1 \gsBefore \gnode_2 \in \ggraph \\
    \gnode_2 = \gsLR{\olbl, \osInStack{\ovalue_1, \ovalue_2, \osMC{Getitem}}, \gtime} \\
    \ovalue_1 = [\omem_0, \ldots, \omem_n] | (\omem_0, \ldots, \omem_n) \\
    0 > \ovalue_2 \;|\; \ovalue > n
    \omstack' = [\osMC{AllocIndexError}, \osMC{Raise}]
  }{
    \gsAddEdge{\gnode_2 \gsBefore \gsLR{\olbl, \osInStack{\omstack'}, \gtime + 1}}
  }
\end{mathpar}
%</graph-binop>

%<*graph-op>
\begin{mathpar}
  \relationRule{Wrong Number of Args}{
    \gnode_1 \gsBefore \gnode_2 \in \ggraph \\
    \gnode_2 = \gsLR{\olbl, \osInStack{\ovalue, \omem_1, \ldots, \omem_n, \opscope, \osMC{Call} \;n}, \gtime} \\
    \ovalue = \omagicm \;|\; \omagicf \\
    \osFunc{CheckArgs}(\ovalue, n) = \osfalse \\
    \omstack' = [\osMC{AllocTypeError}, \osMC{Raise}]
  }{
    \gsAddEdge{\gnode_2 \gsBefore \gsLR{\olbl, \osInStack{\omstack'}, \gtime + 1}}
  }
\end{mathpar}

\begin{mathpar}
  \relationRule{Call}{
    \gnode_1 \gsBefore \gnode_2 \in \ggraph \\
    \gnode_2 = \gsLR{\olbl, \osInStack{\ovalue, \omem, \omem_1, \ldots, \omem_n, \opscope, \osMC{Call} \;n+1}, \gtime} \\
    \ovalue = \osMagic{Call} \\
    \omstack' = [\omem, \osMC{Get}, \ostarvalue{value}, \osMC{Retrieve}, \omem_1, \ldots, \omem_n, \osMC{Alloc}, \osMC{Convert} \;n, \osMC{Advance}]
    % For now, it calls its object's value field.
  }{
    \gsAddEdge{\gnode_2 \gsBefore \gsLR{\olbl, \osInStack{\omstack'}, \gtime + 1}}
  }
\end{mathpar}

\begin{mathpar}
  \relationRule{Type}{
    \gnode_1 \gsBefore \gnode_2 \in \ggraph \\
    \gnode_2 = \gsLR{\olbl, \osInStack{\ovalue, \omem, \opscope, \osMC{Call} \;1}, \gtime} \\
    \ovalue = \osMagic{Type} \\
    \opscope = \gsFunc{Scopeof}(\gnode_2) \\
    \oprogram(\olbl) = \olbl:\oglbl': \ovariable \gteq \oexpr \\
    \omstack' = [\opscope, \omem, \osMC{Get}, \oq \osClown{class} \oq, \osMC{Retrieve}, \ovariable, \osMC{Bind}, \osMC{Advance}]
  }{
    \gsAddEdge{\gnode_2 \gsBefore \gsLR{\olbl, \osInStack{\omstack'}, \gtime + 1}}
  }
\end{mathpar}

\begin{mathpar}
  \relationRule{Bool}{
    \gnode_1 \gsBefore \gnode_2 \in \ggraph \\
    \gnode_2 = \gsLR{\olbl, \osInStack{\ovalue, \omem, \opscope, \osMC{Call} \;1}, \gtime} \\
    \ovalue = \osMagic{Bool} \\
    \omstack' = [\omem, \osMC{Get}, \ostarvalue{value}, \osMC{Retrieve}, \osMC{Get}, \osMC{Bool}, \osMC{Advance}]
  }{
    \gsAddEdge{\gnode_2 \gsBefore \gsLR{\olbl, \osInStack{\omstack'}, \gtime + 1}}
  }
\end{mathpar}

\begin{mathpar}
  \gsCast {IntAdd} {\osMem{IntType}} {\osMem{IntType}, \osMem{FloatType}} {2} {\osMC{Sum}} {} {a} {2}
\end{mathpar}

\begin{mathpar}
  \gsCast {IntNeg} {\osMem{IntType}} {} {} {\osMC{Neg}} {} {a} {1}
\end{mathpar}

\begin{mathpar}
  \gsCast {StrAdd} {\osMem{StrType}} {\osMem{StrType}} {1} {\osMC{StrConcat}} {} {a} {2}
\end{mathpar}

\begin{mathpar}
  \gsCast {StrContains} {\osMem{StrType}} {\osMem{StrType}} {1} {\osMC{StrContains}} {} {a} {2}
\end{mathpar}

\begin{mathpar}
  \gsCast {LstGetitem} {\osMem{LstType}} {\osMem{IntType}} {1} {\osMC{Getitem}} {} {b} {2}
\end{mathpar}

%</graph-op>
