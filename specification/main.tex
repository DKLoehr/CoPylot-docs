\documentclass{article}


\usepackage[usenames,dvipsnames]{xcolor}

% Palette: defunct
% \definecolor{kw}{RGB}{138,43,226}
% \definecolor{func}{RGB}{233,105,0}
% \definecolor{microcode}{RGB}{143,188,187}

% Palette: Nord
% \definecolor{kw}{RGB}{94,129,172}
% \definecolor{func}{RGB}{191,97,106}
% \definecolor{microcode}{RGB}{180,142,173}

% Palette: Solarized
% \definecolor{kw}{HTML}{268BD2} % blue
% \definecolor{func}{HTML}{D33682} % magenta
% \definecolor{microcode}{HTML}{2AA198} % cyan
% \definecolor{microcode}{HTML}{B58900} % yellow
% \definecolor{microcode}{HTML}{D26937} % orange
% \definecolor{microcode}{HTML}{AF005F} % violet

% Palette: Gotham
\definecolor{kw}{HTML}{33859E} % cyan Q
\definecolor{func}{HTML}{C23127} % red Q
% \definecolor{microcode}{HTML}{D26937} % orange
% \definecolor{type}{HTML}{2AA889} % green
% \definecolor{microcode}{HTML}{4E5166} % violet
% \definecolor{magic}{HTML}{888CA6} % magenta
% \definecolor{microcode}{HTML}{EDB443} % yellow

% Palette: Slack
% \definecolor{kw}{HTML}{81D2E0} % blue
\definecolor{microcode}{HTML}{599F8C} % cyan Q
\definecolor{type}{HTML}{CC4876} % magenta Q
% \definecolor{magic}{HTML}{5C3A58} % violet Q
% \definecolor{type}{HTML}{EDB625} % yellow

\definecolor{flame}{RGB}{233,105,0}
\definecolor{gothamorange}{HTML}{D26937} % orange

% Palette: base-16-tomorrow
% @red:             #cc6666; // 08
% @orange:          #de935f; // 09
% @yellow:          #f0c674; // 0A
% @green:           #b5bd68; // 0B
% @cyan:            #8abeb7; // 0C
% @blue:            #81a2be; // 0D
% @purple:          #b294bb; // 0E
% @brown:           #a3685a; // 0F
% \definecolor{kw}{HTML}{81A2BE}
% \definecolor{func}{HTML}{CC6666}
% \definecolor{microcode}{HTML}{8ABEB7}
\definecolor{magic}{HTML}{B294BB}
% \definecolor{type}{HTML}{A3685A}

\usepackage{amsmath}
\usepackage{amssymb}
\usepackage{amsthm}
\usepackage{thmtools}
\usepackage{mathpartir}
\usepackage{stmaryrd}
\usepackage{xifthen}
\usepackage{relsize}
\usepackage{mathtools}
\usepackage{yfonts}
\usepackage{multicol}
\usepackage{hyperref}
\usepackage{catchfilebetweentags}
\usepackage[usenames,dvipsnames]{xcolor}

% Better TT font
\let\oldrmdefault\rmdefault
\let\oldsfdefault\sfdefault
\usepackage{lmodern}
\let\rmdefault\oldrmdefault
\let\sfdefault\oldsfdefault

%%% TeX convenience macros %%%%%%%%%%%%%%%%%%%%%%%%%%%%%%%%%%%%%%%%%%%%%%%%%%%%

% Produces the contents of the second argument only if the first argument is
% empty.
\newcommand{\ifnotempty}[2]{\ifthenelse{\isempty{#1}}{}{#2}}

\newcommand{\znote}[1]{\textit{\color{ForestGreen}(#1 -- ZP)}}
\newcommand{\tnote}[1]{\textit{\color{gothamorange}(#1 -- TC)}}

%%% General algebraic notation %%%%%%%%%%%%%%%%%%%%%%%%%%%%%%%%%%%%%%%%%%%%%%%%

\newcommand{\listConcat}{\ensuremath{\mathop{||}}}
\newcommand{\strConcat}{\ensuremath{\mathop{|}}}

%%% General logic notation %%%%%%%%%%%%%%%%%%%%%%%%%%%%%%%%%%%%%%%%%%%%%%%%%%%%

% Create a relational rule
% [#1] - Additional mathpartir arguments
% {#2} - Name of the rule
% {#3} - Premises for the rule
% {#4} - Conclusions for the rule
\newcommand{\relationRule}[4][]{\inferrule*[lab={\sc #2},#1]{#3}{#4}}

\declaretheorem[within=section]{lemma}
\declaretheorem[name=Definition,numberlike=lemma]{definition}
\declaretheorem[name=Notation,numberlike=lemma]{notation}
\declaretheorem[name=Theorem,numberlike=lemma]{theorem}

%%% Grammar definitions %%%%%%%%%%%%%%%%%%%%%%%%%%%%%%%%%%%%%%%%%%%%%%%%%%%%%%%

% Defines a grammar terminal.
% [#1] - The name of the command to define.  Defaults to "\gt#2".
% {#2} - The text of the grammar terminal.
\newcommand{\defgt}[2][]{%
    \ifthenelse{\isempty{#1}}{\defgt[\defgtname{#2}]{#2}}{%
        \expandafter\newcommand\expandafter{\csname #1\endcsname}{\defgtstyle{#2}}
    }%
}
\newcommand{\defgtstyle}[1]{\mathinner{\texttt{#1}}}
\newcommand{\defgtname}[1]{gt#1}

% Defines a grammar non-terminal.
% {#1} - The name of the command to define.
% {#2} - The math-mode text of the command.
\newcommand{\defgn}[2]{%
    \expandafter\newcommand\expandafter{\csname #1\endcsname}{\defgnstyle{#2}}%
}
\newcommand{\defgnstyle}[1]{\ensuremath{#1}}

% Defines an environment, {grammar}, for displaying language grammars.
% Within this environment, some commands are defined:
%   \grule[description]{non-terminal}{productions}
%   \gor (a symbol for separating productions)
%   \gline (a symbol for breaking a line of productions if more space is needed)
%   \setGrammarVertAdjustment{length} (which sets the vertical adjustment for
%                                      the next grammar newline)
\newcommand{\grammarNoteSpace}{\hspace{4mm}}
\newcounter{grammarnote}
\setcounter{grammarnote}{0}
\newcommand{\grammarDefs}{%
    \global\def\grammarVertAdjustment{0mm}%
    \newcommand{\gcomment}[1]{\hfill%
        \ifnum\value{grammarnote}=0
            \stepcounter{grammarnote}
            \grammarNoteSpace \textrm{\textsmaller{\itshape ##1}}
        \fi
    }%
    \newcommand{\grule}[3][]{%
        \setcounter{grammarnote}{0}
        ##2 & \ifnotempty{##3}{::=} & \newcommand{\gcommenttext}{##1} ##3 \hfill \gcomment{##1} \endgrule
    }%
    \newcommand{\gskip}{&&\endgrule}%
    \def\endgrule{\\[\grammarVertAdjustment]}%
    \newcommand{\gor}{\mathrel{\vert}}%
    \newcommand{\gline}{%
        \hfill \gcomment{\gcommenttext} \\[\grammarVertAdjustment] &&
    }%
}
\def\grammarColPad{\quad}%
\newenvironment{grammar}{%
    \begingroup%
    \grammarDefs%
    \begin{math}\begin{array}{@{}r@{\grammarColPad}c@{\grammarColPad}l@{}}%
}{%
    \end{array}\end{math}%
    \endgroup%
}

%%% CoPylot grammars %%%%%%%%%%%%%%%%%%%%%%%%%%%%%%%%%%%%%%%%%%%%%%%%%%%%%%%%%%

%%% Operational semantics

\defgn{olbl}{\ell}
\defgn{oglbl}{\overset{\star}{\ell}}
\defgn{ovalue}{v}
\defgn{ovariable}{x}
\defgn{odirective}{d}
\defgn{ostmt}{s}
\defgn{oprogram}{S}
\defgn{ostack}{T}
\defgn{ostackframe}{t}
\defgn{obinding}{B}
\defgn{oheap}{H}
\defgn{opscope}{\eta}
\defgn{oparent}{P}
\defgn{omem}{m}
\defgn{ogmem}{\overset{\star}{m}}
\defgn{oexpr}{e}
% Function related
\defgn{ofuncarr}{\rightarrow}
\defgn{omagicf}{\mathfrak{F}}
\defgn{omagicm}{\mathfrak{M}}
\defgn{ogenf}{F}
\defgn{ogenm}{M}
\defgn{osplat}{\ast}
% Code stack related
\defgn{omstack}{Y}
\defgn{omcode}{y}
\defgn{olstack}{Z}
\defgn{olcode}{z}

% These are wrong and should not emerge.
% \defgn{oenv}{E}
% \defgn{ofunc}{fun}
% \newcommand{\opscope}{{\color{red}\eta}}
\newcommand{\oenv}{{\color{red}E}}

\newlength\dunder
\settowidth{\dunder}{\_}

% Special
\newcommand{\osnone}{\texttt{None}}
\newcommand{\ostrue}{\text{\textsmaller{{\color{func}\sc True}}}}
\newcommand{\osfalse}{\text{\textsmaller{{\color{func}\sc False}}}}
% Relations
\newcommand{\osTransition}{\mathrel{\longrightarrow^1}}
\newcommand{\osBefore}[1]{\mathrel{\stackrel{{\scriptscriptstyle\boldsymbol{#1}}}{\lexicallyBefore}}}
% Helpers
\newcommand{\osLookup}{\text{\textsmaller{{\color{func}\sc Lookup}}}}
\newcommand{\osWrap}{\text{\textsmaller{{\color{func}\sc Wrap}}}}
\newcommand{\osGetCall}{\text{\textsmaller{{\color{func}\sc GetCall}}}}
% Format
\newcommand{\osLR}[1]{\langle #1 \rangle}
\newcommand{\osI}[2]{#1 &= #2}
\newcommand{\osInStack}[1]{\olstack \ifthenelse {\equal{#1}{}} {\listConcat} {\ifthenelse {\equal{#1}{\omstack'}}{\listConcat #1 \listConcat}{\listConcat [#1] \listConcat}} \omstack}
% Stylize
\newcommand{\okw}[1]{\;\text{\texttt{{\color{kw}#1}}}\;}
\newcommand{\osMC}[1]{\text{\textsmaller{{\color{microcode}\sc #1}}}}
\newcommand{\osFunc}[1]{\text{\textsmaller{{\color{func}\sc #1}}}}
\newcommand{\osType}[1]{\text{\textsmaller{{\color{type}\sc #1}}}}
\newcommand{\osMagic}[1]{\text{{\color{magic}\sc $\mathfrak{#1}$}}}
% -fix
\newcommand{\ostarvalue}[1]{\star \ovariable_{\texttt{#1}}}
\newcommand{\osInit}[1]{#1_{\text{\textsmaller{{\sc Init}}}}}
\newcommand{\osMem}[1]{\omem_{\texttt{#1}}}
\newcommand{\osClown}[1]{\rule{2\dunder}{0.4pt} #1 \rule{2\dunder}{0.4pt}}
% Magic
\newcommand{\osDuoMagic}[6]{
  % #1: magic name, #2: first type, #3: second type list
  % #4: len(list), #5: MI to call, #6: other things to put before the micro-stack
  \relationRule{#1}{%
    \ovalue = \osMagic{#1} \\%
    \osFunc{CheckArgs}(\ovalue, 2) = \ostrue \\
    \ostack = [\osLR{\opscope, \olbl}] \listConcat \ostack' \\%
    \oprogram(\olbl) = \olbl:\oglbl': \ovariable \gteq \oexpr \\%
    #6%
    \omstack' = [\omem_1, \osMC{Get}, \ostarvalue{value}, \osMC{Retrieve}, \osMC{Get}, #2, \osMC{Assert} \;1, \omem_2, \osMC{Get}, \ostarvalue{value}, \osMC{Retrieve}, \\ \osMC{Get}, \ifthenelse {\equal{#3}{}} {} {#3, \osMC{Assert} \;#4, }\osMC{#5}, \ovariable, \osMC{Assign}, \osMC{Advance}]%
  }{%
    \osInStack{\ovalue, \omem_1, \omem_2, \osMC{Call} \;2}, \ostack, \oheap \osTransition%
    \osInStack{\omstack'}, \ostack, \oheap%
  }%
}

% Used in main
\newcommand{\osend}{\text{\textsmaller{{\color{func}\sc End}}}}
\newcommand{\osBind}{\text{\textsmaller{{\color{func}\sc Bind}}}}
\newcommand{\osMakeObj}{\text{\textsmaller{{\color{func}\sc MakeObj}}}}
\newcommand{\osCatch}{\text{\textsmaller{{\color{func}\sc Catch}}}}

%%% Graph semantics

\defgn{ggraph}{G}
\defgn{gedge}{g}
\defgn{gnode}{a}
\defgn{gtime}{\chi}
\defgn{genter}{\llparenthesis}
\defgn{gleave}{\rrparenthesis}
\defgn{gresult}{R}
\defgn{gvset}{V}

\newcommand{\niton}{\not\owns}

% Special
\newcommand{\gsstart}{\text{\textsmaller{{\color{func}\sc Start}}}}
\newcommand{\gsend}{\text{\textsmaller{{\color{func}\sc End}}}}
% Relations
\newcommand{\gsTransition}{\mathrel{\longrightarrow^1}}
\newcommand{\gsBefore}{\ll}
\newcommand{\gsSkip}{\lll}
\newcommand{\gsPrecede}{\ll \kern -4pt \ll}
% Format
\newcommand{\gsLR}[1]{\langle #1 \rangle}
\newcommand{\gsSet}[1]{\{ #1 \}}
%Stylize
\newcommand{\gsFunc}[1]{\text{\textsmaller{{\color{func}\sc #1}}}}

\newcommand{\gsAddEdge}[1]{\ggraph \gsTransition \ggraph \cup \gsSet{#1}}



%%% Abstract language grammar

\defgt[gtcolon]{:}
\defgt[gteq]{=}

\defgt[gtintplus]{int\ensuremath{{}^{\texttt{+}}}}
\defgt[gtintminus]{int\ensuremath{{}^{\texttt{-}}}}
\defgt[gtintzero]{int\ensuremath{{}^{\texttt{0}}}}

\newcommand{\xstmt}[3]{#1\gtcolon#2\gtcolon#3}

%%% Analysis grammar

\defgn{albl}{\hat{\ell}}
\defgn{avalue}{\hat{v}}
\defgn{avariable}{\hat{x}}
\defgn{adirective}{\hat{d}}
\defgn{astmt}{\hat{s}}
\defgn{aprogram}{\hat{S}}

% this will likely be replaced later when we move to a different kind of answer
\defgn{avalues}{\hat{V}}

\defgn{cfgnode}{\hat{o}}
\defgn{cfgedge}{\hat{g}}
\defgn{cfg}{\hat{G}}

\newcommand{\gstart}{\text{\textsmaller{\sc Start}}}
\newcommand{\gend}{\text{\textsmaller{\sc End}}}

\newcommand{\before}{\mathrel{\mathrlap{<}{\ <}}}
\newcommand{\lexicallyBefore}{\blacktriangleleft}

\defgn{astackelement}{\hat{k}}
\defgn{astack}{\hat{K}}

\newcommand{\kcapture}[1]{\text{\textsmaller{\textsc{Capture}}}(#1)}
\newcommand{\kjump}[1]{\text{\textsmaller{\textsc{Jump}}}(#1)}

%%% CoPylot functions and relations %%%%%%%%%%%%%%%%%%%%%%%%%%%%%%%%%%%%%%%%%%%

\newcommand{\isbefore}{\mathrel{\stackrel{{\scriptscriptstyle\boldsymbol{?}}}{\before}}}
\newcommand{\islexicallyBefore}{\mathrel{\stackrel{{\scriptscriptstyle\boldsymbol{?}}}{\lexicallyBefore}}}

\newcommand{\valueLookup}[3][\cfg]{#1(#2,#3)}

\newcommand{\cfgClosureStep}{\mathrel{\longrightarrow^1}}
\newcommand{\cfgClosureSteps}{\mathrel{\longrightarrow^*}}

\usepackage{enumitem}
\usepackage{caption}


\begin{document}
    \section{CoPylot}

    \subsection{Grammar}



        \begin{figure}\center
              \begin{grammar}
                \grule[variables]{\ovariable}{\osalphanumeric \gor \star \osalphanumeric}
                \grule[labels]{\oglbl}{\olbl \gor \osend}
                \grule[stack]{\ostack}{
                            [\ostackframe,\ldots]
                }
                \grule[stack frames]{\ostackframe}{\oglbl \times \oprogram}
                \grule[programs]{\oprogram}{
                            [\ostmt,\ldots]
                }
                \grule[clauses]{\olbl:\oglbl:\odirective}
                % statements
                \grule[directives]{\odirective}{
                            \ovariable \gteq \oexpr % x = e
                    \gor    \okw{return} \ovariable % return x
                    \gor    \okw{goto} \olbl % goto x
                    \gor    \okw{goto} \olbl \okw{if not} \ovariable % goto x if not v
                    \gline
                    \gor    \okw{raise} \ovariable % raise x
                    \gor    \okw{catch} \ovariable % catch x
                }
                \grule[bindings]{\obinding}{\{\ovariable \mapsto \omem, \ldots\}}
                \grule[heap]{\oheap}{\{\omem \mapsto \ovalue, \ldots\}}
                % value types
                \grule[values]{\ovalue}{
                            \mathbb{Z}
                    % \gor    \osLR{ \omem,\okw{def}(\ovariable) \ofuncarr \oprogram } % < scope, def (x) -> S >
                    % \gor    \osLR{ \omem,\omem,\okw{def}(\ovariable) \ofuncarr \oprogram } % < scope, obj, def (x) -> S >
                    \gor    [\omem, \ldots] % [m, ...]
                    \gor    (\omem, \ldots) % (m, ...)
                    \gor    \obinding % B
                    % \gor    \omagicf
                    % \gor    \osLR{ \omem, \omagicm }
                    \gor    \ogenf
                    \gor    \ogenm
                    \gor    \okw{undefined}
                    \gor    \osnone
                }
                % expressions
                \grule[expressions]{\oexpr}{
                            \ovalue
                    \gor    \ovariable % x
                    \gor    \okw{def} \ovariable(\ovariable,\ldots) \gteq \{ \oprogram \}
                            % def x(x, ...) = {S}
                    \gor    \ovariable(\ovariable, \ldots) % x(x, ...)
                    \gor    [\ovariable, \ldots] % [x, ...]
                    \gor    (\ovariable, \ldots) % (x, ...)
                }
                \grule[parental map]{\oparent}{\omem \mapsto \omem}
                \grule[memory locations]{\omem}{}
                \grule[general functions]{\ogenf}{
                            \osLR{ \omem,\okw{def}(\ovariable) \ofuncarr \oprogram } % < scope, def (x) -> S >
                    \gor    \omagicf
                }
                \grule[general methods]{\ogenm}{
                            \osLR{ \omem,\omem,\okw{def}(\ovariable) \ofuncarr \oprogram } % < scope, obj, def (x) -> S >
                    \gor    \osLR{ \omem, \omagicm }
                }
                \grule[magic functions]{\omagicf}{}
                \grule[magic methods]{\omagicm}{}

              \end{grammar}

              \begin{definition}
                $$
                  \osBind (\oheap,\omem_0,\ovariable,\omem) = \oheap' \text{ such that } \obinding = \oheap[\omem_0], \obinding' = \obinding[\ovariable \mapsto \omem], \oheap' = \oheap[\omem_0 \mapsto \obinding']
                $$
              \end{definition}

              \begin{definition}
                \begin{flalign*}
                  \osBind (\oheap,\omem_0,\ovariable_1,\omem_1,\ldots,\ovariable_n,\omem_n) \\
                  &= \osBind (\ldots ((\osBind ((\osBind (\oheap, \omem_0, \ovariable_1, \omem_1)), \omem_0, \ovariable_2, \omem_2), \ldots), \omem_0, \ovariable_n, \omem_n)
                \end{flalign*}
              \end{definition}

              % TODO: add lookup rule
              \begin{definition}
                $$
                  \osLookup(\omem_0,\oparent,\oheap,\ovariable_2) =
                $$
              \end{definition}

              \begin{definition}
                $\olbl:\olbl':\odirective \in \oprogram
                , \olbl':\olbl'':\odirective' \in \oprogram
                , \omem' = \oparent[\omem]$
                  \begin{equation}
                      \osCatch (\osLR{\oparent,\olbl,\oprogram,\omem}) =
                      \begin{cases}
                        \olbl', \ovariable, \omem' & \text{if}\ \odirective' = \okw{catch} \ovariable \\
                        \okw{undefined}, \omem' & \text{otherwise}
                      \end{cases}
                    \end{equation}
              \end{definition}

              \begin{definition}
                  \begin{equation}
                    \osGetObj (\oheap, \omem) =
                      \begin{cases}
                        \obinding, & \text{if}\ \ovalue = \obinding \\
                        \obinding = {\ostarvalue{value} \mapsto \ovalue}, & \text{otherwise}
                      \end{cases}
                      , \oheap[\omem] = \ovalue
                    \end{equation}
              \end{definition}

              \begin{definition}
                $\oheap[\omem][\text{\_\_call\_\_}] = \omem'
                , \oheap[\omem'] = \ovalue$
                  \begin{equation}
                    % \begin{aligned}
                    \osGetCall (\oheap, \omem) =
                      \begin{cases}
                        \osGetCall (\oheap, \omem'), & \text{if }\ \ovalue = \obinding \\
                        \omem',
                        % & \parbox[t]{.4\textwidth}
                        % {$\text{if } \ovalue = \osLR{\omem_0,\okw{def}(\ovariable_1,\ldots,\ovariable_n) \ofuncarr \oprogram} \; | \;
                        % \osLR{\omem_0,\omem_{\texttt{obj}},\okw{def}(\ovariable_1,\ldots,\ovariable_n) \ofuncarr \oprogram} \; | \;
                        % \osLR{\omagicf} \; | \;
                        % \osLR{\omem_{\texttt{obj}},\omagicm}$}
                        & \text{if } \ovalue = \ogenf \;|\; \ogenm
                      \end{cases}
                    % \end{aligned}
                    \end{equation}
              \end{definition}

            \caption{Operational Semantics}
            \label{fig_languageOS}
            \end{figure}

            \begin{figure}\center
              \ContinuedFloat

              \begin{mathpar}
              \relationRule{Literal Assignment}{
                  \oprogram(\olbl) = \olbl:\olbl':\ovariable = \ovalue \\
                  \obinding_{\texttt{obj}} = \{\ostarvalue{value} \mapsto \omem\} \\
                  \oheap' = \oheap[\omem \mapsto \ovalue, \omem' \mapsto \obinding_{\texttt{obj}}] \\
                  \omem,\omem' \notin \oheap \\
                  \osBind (\oheap',\omem_0,\ovariable,\omem') = \oheap'' \\
                  \olbl \osBefore{S} \oglbl''
              }{
                  [\osLR{ \olbl,\oprogram }] \listConcat \ostack,\oheap,\oparent,\omem_0 \osTransition
                  [\osLR{ \oglbl'',\oprogram }] \listConcat \ostack,\oheap'',\oparent,\omem_0
              }

              \relationRule{Name Assignment}{
                  \oprogram(\olbl) = \olbl:\olbl':\ovariable_1 = \ovariable_2 \\
                  \omem = \osLookup (\omem_0,\oparent,\oheap,\ovariable_2)\\
                  % \obinding = \oenv(\omem_0) \\
                  % \obinding' = \obinding[\ovariable_1 \mapsto \omem] \\
                  % \oenv' = \oenv[\omem_0 \mapsto \obinding'] \\
                  \osBind (\oheap,\omem_0,\ovariable_1,\omem) = \oheap' \\
                  \olbl \osBefore{S} \oglbl''
              }{
                  [\osLR{ \olbl,\oprogram }] \listConcat \ostack,\oheap,\oparent,\omem_0 \osTransition
                  [\osLR{ \oglbl'',\oprogram }] \listConcat \ostack,\oheap',\oparent,\omem_0
              }

              \relationRule{List Assignment}{
                  \oprogram(\olbl) = \olbl:\olbl':\ovariable = [\ovariable_1,\ldots,\ovariable_n] \\
                  \forall i \in \{1,\ldots,n\}, \omem_i = \osLookup(\omem_0,\oparent,\oheap,\ovariable_i) \\
                  \ovalue = [\omem_1,\ldots,\omem_n] \\
                  \obinding_{\texttt{obj}} = \{\ostarvalue{value} \mapsto \omem\} \\
                  \oheap' = \oheap[\omem \mapsto \ovalue, \omem' \mapsto \obinding_{\texttt{obj}}] \\
                  \omem,\omem' \notin \oheap \\
                  % \obinding = \oenv(\omem_0) \\
                  % \obinding' = \obinding[\ovariable \mapsto \omem'] \\
                  % \oenv' = \oenv[\omem_0 \mapsto \obinding'] \\
                  \osBind (\oheap',\omem_0,\ovariable,\omem') = \oheap'' \\
                  \olbl \osBefore{S} \oglbl''
              }{
                  \osLR{ \olbl,\oprogram }] \listConcat \ostack,\oheap,\oparent,\omem_0 \osTransition
                  [\osLR{ \oglbl'',\oprogram }] \listConcat \ostack,\oheap'',\oparent,\omem_0
              }

              \relationRule{Tuple Assignment}{
                  \oprogram(\olbl) = \olbl:\olbl':\ovariable = (\ovariable_1,\ldots,\ovariable_n) \\
                  \forall i \in \{1,\ldots,n\}, \omem_i = \osLookup(\omem_0,\oparent,\oheap,\ovariable_i) \\
                  \ovalue = (\omem_1,\ldots,\omem_n) \\
                  \obinding_{\texttt{obj}} = \{\ostarvalue{value} \mapsto \omem\} \\
                  \oheap' = \oheap[\omem \mapsto \ovalue, \omem' \mapsto \obinding_{\texttt{obj}}] \\
                  \omem,\omem' \notin \oheap \\
                  % \obinding = \oenv(\omem_0) \\
                  % \obinding' = \obinding[\ovariable \mapsto \omem'] \\
                  % \oenv' = \oenv[\omem_0 \mapsto \obinding'] \\
                  \osBind (\oheap',\omem_0,\ovariable,\omem') = \oheap'' \\
                  \olbl \osBefore{S} \oglbl''
              }{
                  \osLR{ \olbl,\oprogram }] \listConcat \ostack,\oheap,\oparent,\omem_0 \osTransition
                  [\osLR{ \oglbl'',\oprogram }] \listConcat \ostack,\oheap'',\oparent,\omem_0
              }

              \relationRule{Function Assignment}{
                  \oprogram(\olbl) = \olbl:\olbl': \ovariable_1 = \okw{def} (\ovariable_2, \ldots, \ovariable_n) = \{\oprogram'\} \\
                  \ovalue = \osLR{ \omem_0, \okw{def}(\ovariable_2, \ldots, \ovariable_n) \ofuncarr \oprogram' } \\
                  \obinding_{\texttt{obj}} = \{\ostarvalue{value} \mapsto \omem, \text{\_\_call\_\_} \mapsto \osLR{\omem', \omagicm_{\texttt{call}}} \} \\
                  \oheap' = \oheap[\omem \mapsto \ovalue, \omem' \mapsto \obinding_{\texttt{obj}}] \\
                  \omem,\omem' \notin \oheap \\
                  % \obinding = \oenv(\omem_0) \\
                  % \obinding' = \obinding[\ovariable_1 \mapsto \omem] \\
                  % \oenv' = \oenv[\omem_0 \mapsto \obinding'] \\
                  \osBind (\oheap',\omem_0,\ovariable_1,\omem') = \oheap'' \\
                  \olbl \osBefore{S} \oglbl''
              }{
                  [\osLR{ \olbl,\oprogram }] \listConcat \ostack,\oheap,\oparent,\omem_0 \osTransition
                  [\osLR{ \oglbl'',\oprogram }] \listConcat \ostack,\oheap'',\oparent,\omem_0
              }

              \relationRule{Function Call Assignment}{
                  \oprogram(\olbl) = \olbl:\olbl':\ovariable_1 = \ovariable_2(\ovariable_3,\ldots,\ovariable_n) \\
                  \omem_{\texttt{raw}} = \osLookup (\omem_0,\oparent,\oheap,\ovariable_2)\\
                  \omem = \osGetCall(\oheap,\omem_{\texttt{raw}}) \\
                  \oheap[\omem][\ostarvalue{obj}] = \osLR{ \omem_0',\okw{def}(\ovariable_3',\ldots,\ovariable_n') \ofuncarr \oprogram' } \\
                  \omem_0'' \notin \oheap \\
                  \oheap' = \oheap[\omem_0'' \mapsto \{\;\}] \\
                  \forall i, 3 \leq i \leq n, \omem_i' = \osLookup (\omem_0,\oparent,\oheap,\ovariable_i) \\
                  % \obinding' = \obinding [\ovariable_4 \mapsto \omem'] \\
                  % \oenv' = \oenv[\omem_0'' \mapsto \obinding'] \\
                  \osBind (\oheap',\omem_0'',\ovariable_1',\omem_1',\ldots,\ovariable_n',\omem_n') = \oheap'' \\
                  \oparent' = \oparent \cup \{\omem_0'' \mapsto \omem_0'\} \\
                  \oprogram' = [\olbl'':\olbl''':\odirective, \ldots]
              }{
                  [\osLR{ \olbl,\oprogram }] \listConcat \ostack,\oheap,\oparent,\omem_0 \osTransition
                  [\osLR{ \olbl'',\oprogram' },\osLR{ l,S }] \listConcat \ostack,\oheap'',\oparent',\omem_0''
              }

              \tnote{Real python functions don't work as above. They just run the code. Fix it later.}

            \end{mathpar}
            \caption{Operational Semantics (cont.)}
            \label{fig_languageOS}
        \end{figure}

        \begin{figure}\center
          \ContinuedFloat
            \begin{mathpar}

              \relationRule{Method Call Assignment}{
                  \oprogram(\olbl) = \olbl:\olbl':\ovariable_1 = \ovariable_2(\ovariable_3,\ldots,\ovariable_n) \\
                  \omem_{\texttt{raw}} = \osLookup (\omem_0,\oparent,\oheap,\ovariable_2)\\
                  \omem = \osGetCall(\oheap,\omem_{\texttt{raw}}) \\
                  \oheap[\omem][\ostarvalue{obj}] = \osLR{ \omem_0',\omem_{\texttt{obj}}, \okw{def}(\ovariable_3',\ldots,\ovariable_n') \ofuncarr \oprogram' } \\
                  \omem_0'' \notin \oheap \\
                  \oheap' = \oheap[\omem_0'' \mapsto \{\ovariable_{\texttt{self}} \mapsto \omem_\texttt{obj}\}] \\
                  \forall i, 3 \leq i \leq n, \omem_i' = \osLookup (\omem_0,\oparent,\oheap,\ovariable_i) \\
                  \osBind  (\oheap', \omem_0'', \ovariable_1', \omem_1', \ldots, \ovariable_n', \omem_n') = \oheap'' \\
                  \oparent' = \oparent \cup \{\omem_0'' \mapsto \omem_0'\} \\
                  \oprogram' = [\olbl'':\olbl''':\odirective, \ldots]
              }{
                  [\osLR{ \olbl,\oprogram }] \listConcat  \ostack,\oheap,\oparent,\omem_0 \osTransition
                  [\osLR{ \olbl'',\oprogram' },\osLR{ l,S }] \listConcat \ostack,\oheap'',\oparent',\omem_0''
              }

              \relationRule{Attribute Assignment}{
                  \oprogram(\olbl) = \olbl:\olbl':\ovariable_1 = \ovariable_2.\ovariable_3 \\
                  \omem = \osLookup(\omem_0,\oparent,\oheap,\ovariable_2) \\
                  \oheap[\omem] = \obinding \\
                  \obinding[\ovariable_3] = \omem' \\
                  % \omem' \notin \oheap \\
                  % \oheap[\omem'] = \ovalue \\
                  % TODO: determine whether is object, wrap accordingly
                  \obinding_{\texttt{obj}} = \osGetObj (\oheap, \omem') \\
                  \omem'' \notin \oheap \\
                  \oheap' = \oheap[\omem'' \mapsto \obinding_{\texttt{obj}}] \\
                  \osBind (\oheap',\omem_0,\ovariable_1,\omem'') = \oheap'' \\
                  \olbl \osBefore{S} \oglbl'
              }{
                  [\osLR{ \olbl,\oprogram }] \listConcat \ostack,\oheap,\oparent,\omem_0 \osTransition
                  [\osLR{ \oglbl',\oprogram }] \listConcat \ostack,\oheap'',\oparent,\omem_0
              }

              \tnote{Gets values wrapped in objects.}

              \relationRule{Raise Exception Caught}{
                  \oprogram(\olbl) = \olbl:\olbl': \okw{raise} \ovariable \\
                  \ostack = [\osLR{\olbl_1,\oprogram_1},\ldots,\osLR{\olbl_n,\oprogram_n}] \\
                  % \olbl_1 = \olbl' \\
                  \omem_0^0 = \omem_0 \\
                  \forall i, 1 \leq i \leq k, k < n, \osCatch (\osLR{\oparent,\olbl_i,\oprogram_i,\omem_0^{i-1}}) = \osLR{\okw{undefined}, \okw{undefined}, \omem_0^i} \\
                  \osCatch(\osLR{\oparent,\olbl_{k+1},\oprogram_{k+1},\omem_0^k}) = \osLR{\olbl', \ovariable', \omem_0'} \\
                  \omem = \osLookup(\omem_0',\oparent,\oheap,\ovariable) \\
                  \osBind(\oheap,\omem_0,\ovariable',\omem) = \oheap' \\
                  \olbl_{s}' \osBefore{S_{k+1}} \oglbl''
              }{
                  [\osLR{ \olbl,\oprogram }] \listConcat \ostack,\oheap,\oparent,\omem_0 \osTransition
                  [\osLR{ \oglbl'',\oprogram_{k+1} }] \listConcat \ostack,\oheap',\oparent,\omem_0'
              }

              \relationRule{Raise Exception Escaped}{
                  \oprogram(\olbl) = \olbl:\olbl': \okw{raise} \ovariable \\
                  \ostack = [\osLR{\olbl_1,\oprogram_1},\ldots,\osLR{\olbl_n,\oprogram_n}] \\
                  \olbl_1 = \olbl' \\
                  \forall i, 1 \leq i \leq n, \osCatch (\osLR{\oparent,\olbl_i,\oprogram_i,\omem_0^{i-1}}) = \osLR{\okw{undefined}, \okw{undefined}, \omem_0^i} \\
              }{
                  [\osLR{ \olbl,\oprogram }] \listConcat \ostack,\oheap,\oparent,\omem_0 \osTransition
                  [\;],\oheap,\oparent,\omem_0
              }

              \relationRule{Pass}{
                  \oprogram(\olbl) = \olbl:\olbl': \okw{pass} \\
                  \olbl \osBefore{S} \oglbl''
              }{
                  [\osLR{ \olbl,\oprogram }] \listConcat \ostack,\oheap,\oparent,\omem_0 \osTransition
                  [\osLR{ \oglbl'',\oprogram }] \listConcat \ostack,\oheap,\oparent,\omem_0
              }

              % \relationRule{Return}{
              %     \oprogram(\olbl) = \olbl:\olbl' : \okw{return}  \\
              %     \ostack = [\ostackframe,\osLR{ \olbl'',\oprogram' }] \listConcat \ostack' \\
              %     \omem_0' = \oparent[\omem_0] \\
              %     \olbl'' \osBefore{S'} \oglbl''' \\
              % }{
              %     [\ostackframe,\osLR{ \olbl'',\oprogram' }] \listConcat \ostack,\oheap,\oparent,\omem_0 \osTransition
              %     [\osLR{ \oglbl''',\oprogram' }] \listConcat \ostack',\oheap,\oparent,\omem_0'
              % }

              % Used to be "return with arguments"
              \relationRule{Return}{
                  \oprogram(\olbl) = \olbl:\olbl' : \okw{return}  \ovariable \\
                  \ostack = [\ostackframe,\osLR{ \olbl'',\oprogram' }] \listConcat \ostack' \\
                  \omem = \osLookup(\omem_0,\oparent,\oheap,\ovariable) \\
%
                  \omem_0' = \oparent[\omem_0] \\
                  % \obinding = \oenv(\omem_0') \\
                  % \obinding' = \obinding[\ovariable_1 \mapsto \omem] \\
                  % \oenv' = \oenv[\omem_0' \mapsto \obinding'] \\
                  \oprogram'(\olbl'') = \olbl'':\olbl''':\ovariable_1 = \oexpr \\
                  \osBind (\oheap,\omem_0',\ovariable_1,\omem) = \oheap' \\
%
                  \olbl'' \osBefore{S'} \oglbl'''' \\
              }{
                  [\ostackframe,\osLR{ \olbl'',\oprogram' }] \listConcat \ostack',\oheap,\oparent,\omem_0 \osTransition
                  [\osLR{ \oglbl'''',\oprogram' }] \listConcat \ostack',\oheap',\oparent,\omem_0'
              }

        \end{mathpar}
        \caption{Operational Semantics (cont.)}
        \label{fig_languageOS}
    \end{figure}

    \begin{figure}\center
      \ContinuedFloat
        \begin{mathpar}

              \relationRule{Goto}{
                  \oprogram(\olbl) = \olbl:\olbl' : \okw{goto} \olbl'' \\
                  (\olbl'':\olbl''':\odirective) \in \oprogram
              }{
                  [\osLR{ \olbl,\oprogram }] \listConcat \ostack,\oheap,\oparent,\omem_0 \osTransition
                  [\osLR{ \olbl'',\oprogram }] \listConcat \ostack,\oheap,\oparent,\omem_0
              }

              \relationRule{GotoIfNot}{
                  \oprogram(\olbl) = \olbl:\olbl' : \okw{goto} \olbl'' \okw{if not} \ovariable \\
                  \omem = \osLookup(\omem_0,\oparent,\oheap,\ovariable) \\
                  \oheap[\omem][\ostarvalue{value}] = \osfalse \\
                  (\olbl'':\olbl''':\odirective) \in \oprogram
              }{
                  [\osLR{ \olbl,\oprogram }] \listConcat \ostack,\oheap,\oparent,\omem_0 \osTransition
                  [\osLR{ \olbl'',\oprogram }] \listConcat \ostack,\oheap,\oparent,\omem_0
              }

            \relationRule{GotoIfNot Failed}{
                \oprogram(\olbl) = \olbl:\olbl' : \okw{goto} \olbl'' \okw{if not} \ovariable \\
                \omem = \osLookup(\omem_0,\oparent,\oheap,\ovariable) \\
                \oheap[\omem][\ostarvalue{value}] = \ostrue \\
                \olbl \osBefore{S'} \oglbl'' \\
            }{
                [\osLR{ \olbl,\oprogram }] \listConcat \ostack,\oheap,\oparent,\omem_0 \osTransition
                [\osLR{ \oglbl'',\oprogram }] \listConcat \ostack,\oheap,\oparent,\omem_0
            }

            \relationRule{Name Statement}{
                \oprogram(\olbl) = \olbl:\olbl' : \oexpr \\
                \obinding = \oheap[\omem_0] \\
                \forall \ovariable \in \oexpr, \exists \obinding[\ovariable] \\
                %%% \ovariable | \ovalue | \ovariable(\ovariable_1,\ldots,\ovariable_n) | \ovariable.\ovariable' | [\ovariable_1,\ldots,\ovariable_n] | (\ovariable_1,\ldots,\ovariable_n)
                \olbl \osBefore{S'} \oglbl'' \\
            }{
                [\osLR{ \olbl,\oprogram }] \listConcat \ostack,\oheap,\oparent,\omem_0 \osTransition
                [\osLR{ \oglbl'',\oprogram }] \listConcat \ostack,\oheap,\oparent,\omem_0
            }

            \relationRule{End of Function}{
                \ostack = [\osLR{ \osend,\oprogram },\osLR{ \olbl,\oprogram' }]\listConcat \ostack' \\
                \omem_0' = \oparent[\omem_0] \\
                \oprogram'(\olbl) = \olbl:\olbl'': \ovariable = \oexpr \\
                \omem \notin \oheap \\
                \oheap' = \oheap [\omem \mapsto \osnone] \\
                \osBind (\oheap',\omem_0',\ovariable,\omem) = \oheap'' \\
                \olbl \osBefore{S'} \oglbl' \\
            }{
                [\osLR{ \osend,\oprogram },\osLR{ \olbl,\oprogram' }] \listConcat \ostack',\oheap,\oparent,\omem_0 \osTransition
                [\osLR{ \oglbl',\oprogram' }] \listConcat \ostack',\oheap'',\oparent,\omem_0'
            }

            \relationRule{End of Program}{
                \ostack = [\osLR{\osend,\oprogram}] \\
            }{
                \ostack,\oheap,\oparent,\omem_0 \osTransition
                [\;],\oheap,\oparent,\omem_0
            }
        \end{mathpar}
        \caption{Operational Semantics (cont.)}
        \label{fig_languageOS}
    \end{figure}

    \begin{figure}\center
        \begin{grammar}
            \grule[abstract programs]{\aprogram}{
                        [\astmt,\ldots]
            }
            \grule[abstract statements]{\astmt}{
                        \xstmt{\albl}{\albl}{\adirective}
            }
            \grule[abstract directives]{\adirective}{
                        \avariable \gteq \avalue
                \gor    \avariable \gteq \avariable
            }
            \grule[abstract values]{\avalue}{
                        \gtintplus
                \gor    \gtintminus
                \gor    \gtintzero
            }
            \grule[abstract variables]{\avariable}{}
            \grule[abstract labels]{\albl}{}

        \end{grammar}
        \caption{Normalized Python Language Grammar}
        \label{fig_languageGrammar}
    \end{figure}

    The grammar of the language to be analyzed appears in Figure~\ref{fig_languageGrammar}.

    We assume throughout the rest of this document that a fixed program $\aprogram$ is under analysis.  \znote{TODO: describe here the idea of a bijection between labels and statements in this fixed program.}

    \subsection{Control Flow}

    The grammar of control flow graphs appears in Figure~\ref{fig_cfgGrammar}.  \znote{Discuss construction of initial graph.}

    \begin{figure}\center
        \begin{grammar}
            \grule[control flow graphs]{\cfg}{
                        \{\cfgedge, \ldots\}
            }
            \grule[control flow graph edge]{\cfgedge}{
                        \cfgnode \lexicallyBefore \cfgnode
                \gor    \cfgnode \before \cfgnode
            }
            \grule[control flow graph nodes]{\cfgnode}{
                        \gstart
                \gor    \gend
                \gor    \astmt
            }
        \end{grammar}
        \caption{Control Flow Graph Grammar}
        \label{fig_cfgGrammar}
    \end{figure}



    We write $\cfgnode \islexicallyBefore \cfgnode'$ to denote $(\cfgnode \lexicallyBefore \cfgnode' \in \cfg$ when $\cfg$ is understood from context.  Likewise, we write $\cfgnode \isbefore \cfgnode'$ to denote $(\cfgnode \before \cfgnode' \in \cfg$ when $\cfg$ is understood from context.

    We define a relation $\cfgClosureStep$ to perform control flow graph closure.

    \begin{definition}
        Let $\cfg \cfgClosureStep \cfg'$ be the least relation satisfying the rules appearing in Figure~\ref{fig_cfgClosure}.  Throughout these rules, the predicates $\isbefore$ and $\islexicallyBefore$ refer to graph $\cfg$.
    \end{definition}

    \begin{figure}
        \begin{mathpar}
            \relationRule{Lexical Start}{
                \gstart \islexicallyBefore \cfgnode
            }{
                \cfg \cfgClosureStep \cfg \cup \{\gstart \before \cfgnode\}
            }

            \relationRule{Literal Assignment}{
                \cfgnode_1 = (\avariable \gteq \avalue) \\
                \cfgnode_1 \islexicallyBefore \cfgnode_2
            }{
                \cfg \cfgClosureStep \cfg \cup \{\cfgnode_1 \before \cfgnode_2\}
            }

            \relationRule{Variable Accessible}{
                \cfgnode_1 = (\avariable \gteq \avariable') \\
                \cfgnode_1 \islexicallyBefore \cfgnode_2 \\
                \avalue \in \valueLookup{\cfgnode_1}{[\avariable']} \\
                \avalue \neq {\text{\textsmaller{\sc Undefined}}}
            }{
                \cfg \cfgClosureStep \cfg \cup \{\cfgnode_1 \before \cfgnode_2\}
            }
        \end{mathpar}
        \caption{Control Flow Graph Closure}
        \label{fig_cfgClosure}
    \end{figure}

    \subsection{Value Lookup}

    The value lookup function uses the additional grammar in Figure~\ref{fig_valueLookupGrammar}.

    \begin{figure}
        \begin{grammar}
            \grule[lookup stacks]{\astack}{
                        [\astackelement, \ldots]
            }
            \grule[lookup stack elements]{\astackelement}{
                        \avariable
                \gor    \avalue % subject to change to e.g. heap fragments
                \gor    \kcapture{\mathbb{N}}
                \gor    \kjump{\cfgnode}
            }
        \end{grammar}
        \caption{Value Lookup Grammar}
        \label{fig_valueLookupGrammar}
    \end{figure}

    \begingroup
        \newenvironment{enumerateClauses}%
            {\begin{enumerate}[label=(\alph*),ref=\arabic{enumi}\alph*]}%
            {\end{enumerate}}
        \newenvironment{enumerateSubclauses}%
            {\begin{enumerate}[label=\roman*.,ref=\arabic{enumi}\alph{enumii}(\roman*)]}%
            {\end{enumerate}}
        \newcommand{\clauseSectionTitle}[1]{\textbf{#1}}
        \newcommand{\clauseSubsectionTitle}[1]{\underline{\smash{#1}}}
        \newcommand{\rulename}[1]{%
            \begingroup%
                \setlength{\fboxsep}{1.5pt}%
                \fcolorbox{black}{gray!15!white}{%
                    \textsc{\textsmaller{#1}}%
                }%
                \vspace*{1pt}%
            \endgroup%
            \\%
        }
        \begin{definition}
            Given a control-flow graph $\cfg$, let $\valueLookup[\cfg]{\cfgnode_0}{\astack}$ be the function returning the least set $\avalues$ which satisfies the following conditions:

            \begin{enumerate}
                \item \clauseSectionTitle{Value Manipulation}
                \begin{enumerateClauses}
                    \item \rulename{Result}
                        If
                            $\astack = [\avalue]$,
                        then
                            $\avalue \in \avalues$.
                \end{enumerateClauses}

                \item \clauseSectionTitle{Variable Lookup}
                \begin{enumerateClauses}
                    \item \rulename{Value Discovery}
                        If
                            $\cfgnode_1 \isbefore \cfgnode_0$,
                            $\cfgnode_1 =
                                \xstmt{\albl_1}{\albl_2}{\avariable \gteq \avalue}$, and
                            $\astack = [\avariable] \listConcat \astack'$,
                        then
                            $\valueLookup{\cfgnode_1}{[\avalue] \listConcat \astack'} \subseteq \avalues$.

                    \item \rulename{Value Skip}
                        If
                            $\cfgnode_1 \isbefore \cfgnode_0$,
                            $\cfgnode_1 =
                                \xstmt{\albl_1}{\albl_2}{\avariable' \gteq \avalue}$,
                            $\astack = [\avariable] \listConcat \astack'$, and
                            $\avariable \neq \avariable'$,
                        then
                            $\valueLookup{\cfgnode_1}{\astack} \subseteq \avalues$.
                    \item \rulename{Value Aliasing}
                        If
                            $\cfgnode_1 \isbefore \cfgnode_0$,
                            $\cfgnode_1 =
                                \xstmt{\albl_1}{\albl_2}{\avariable \gteq \avariable'}$, and
                            $\astack = [\avariable] \listConcat \astack'$,
                        then
                            $\valueLookup{\cfgnode_1}{[\avariable'] \listConcat \astack} \subseteq \avalues$.
                \end{enumerateClauses}


            \end{enumerate}
        \end{definition}
        \endgroup
\end{document}
