\documentclass{article}


\usepackage[usenames,dvipsnames]{xcolor}

% Palette: defunct
% \definecolor{kw}{RGB}{138,43,226}
% \definecolor{func}{RGB}{233,105,0}
% \definecolor{microcode}{RGB}{143,188,187}

% Palette: Nord
% \definecolor{kw}{RGB}{94,129,172}
% \definecolor{func}{RGB}{191,97,106}
% \definecolor{microcode}{RGB}{180,142,173}

% Palette: Solarized
% \definecolor{kw}{HTML}{268BD2} % blue
% \definecolor{func}{HTML}{D33682} % magenta
% \definecolor{microcode}{HTML}{2AA198} % cyan
% \definecolor{microcode}{HTML}{B58900} % yellow
% \definecolor{microcode}{HTML}{D26937} % orange
% \definecolor{microcode}{HTML}{AF005F} % violet

% Palette: Gotham
\definecolor{kw}{HTML}{33859E} % cyan Q
\definecolor{func}{HTML}{C23127} % red Q
% \definecolor{microcode}{HTML}{D26937} % orange
% \definecolor{type}{HTML}{2AA889} % green
% \definecolor{microcode}{HTML}{4E5166} % violet
% \definecolor{magic}{HTML}{888CA6} % magenta
% \definecolor{microcode}{HTML}{EDB443} % yellow

% Palette: Slack
% \definecolor{kw}{HTML}{81D2E0} % blue
\definecolor{microcode}{HTML}{599F8C} % cyan Q
\definecolor{type}{HTML}{CC4876} % magenta Q
% \definecolor{magic}{HTML}{5C3A58} % violet Q
% \definecolor{type}{HTML}{EDB625} % yellow

\definecolor{flame}{RGB}{233,105,0}
\definecolor{gothamorange}{HTML}{D26937} % orange

% Palette: base-16-tomorrow
% @red:             #cc6666; // 08
% @orange:          #de935f; // 09
% @yellow:          #f0c674; // 0A
% @green:           #b5bd68; // 0B
% @cyan:            #8abeb7; // 0C
% @blue:            #81a2be; // 0D
% @purple:          #b294bb; // 0E
% @brown:           #a3685a; // 0F
% \definecolor{kw}{HTML}{81A2BE}
% \definecolor{func}{HTML}{CC6666}
% \definecolor{microcode}{HTML}{8ABEB7}
\definecolor{magic}{HTML}{B294BB}
% \definecolor{type}{HTML}{A3685A}

\usepackage{amsmath}
\usepackage{amssymb}
\usepackage{amsthm}
\usepackage{thmtools}
\usepackage{mathpartir}
\usepackage{stmaryrd}
\usepackage{xifthen}
\usepackage{relsize}
\usepackage{mathtools}
\usepackage{yfonts}
\usepackage{multicol}
\usepackage{hyperref}
\usepackage{catchfilebetweentags}
\usepackage[usenames,dvipsnames]{xcolor}

% Better TT font
\let\oldrmdefault\rmdefault
\let\oldsfdefault\sfdefault
\usepackage{lmodern}
\let\rmdefault\oldrmdefault
\let\sfdefault\oldsfdefault

%%% TeX convenience macros %%%%%%%%%%%%%%%%%%%%%%%%%%%%%%%%%%%%%%%%%%%%%%%%%%%%

% Produces the contents of the second argument only if the first argument is
% empty.
\newcommand{\ifnotempty}[2]{\ifthenelse{\isempty{#1}}{}{#2}}

\newcommand{\znote}[1]{\textit{\color{ForestGreen}(#1 -- ZP)}}
\newcommand{\tnote}[1]{\textit{\color{gothamorange}(#1 -- TC)}}

%%% General algebraic notation %%%%%%%%%%%%%%%%%%%%%%%%%%%%%%%%%%%%%%%%%%%%%%%%

\newcommand{\listConcat}{\ensuremath{\mathop{||}}}
\newcommand{\strConcat}{\ensuremath{\mathop{|}}}

%%% General logic notation %%%%%%%%%%%%%%%%%%%%%%%%%%%%%%%%%%%%%%%%%%%%%%%%%%%%

% Create a relational rule
% [#1] - Additional mathpartir arguments
% {#2} - Name of the rule
% {#3} - Premises for the rule
% {#4} - Conclusions for the rule
\newcommand{\relationRule}[4][]{\inferrule*[lab={\sc #2},#1]{#3}{#4}}

\declaretheorem[within=section]{lemma}
\declaretheorem[name=Definition,numberlike=lemma]{definition}
\declaretheorem[name=Notation,numberlike=lemma]{notation}
\declaretheorem[name=Theorem,numberlike=lemma]{theorem}

%%% Grammar definitions %%%%%%%%%%%%%%%%%%%%%%%%%%%%%%%%%%%%%%%%%%%%%%%%%%%%%%%

% Defines a grammar terminal.
% [#1] - The name of the command to define.  Defaults to "\gt#2".
% {#2} - The text of the grammar terminal.
\newcommand{\defgt}[2][]{%
    \ifthenelse{\isempty{#1}}{\defgt[\defgtname{#2}]{#2}}{%
        \expandafter\newcommand\expandafter{\csname #1\endcsname}{\defgtstyle{#2}}
    }%
}
\newcommand{\defgtstyle}[1]{\mathinner{\texttt{#1}}}
\newcommand{\defgtname}[1]{gt#1}

% Defines a grammar non-terminal.
% {#1} - The name of the command to define.
% {#2} - The math-mode text of the command.
\newcommand{\defgn}[2]{%
    \expandafter\newcommand\expandafter{\csname #1\endcsname}{\defgnstyle{#2}}%
}
\newcommand{\defgnstyle}[1]{\ensuremath{#1}}

% Defines an environment, {grammar}, for displaying language grammars.
% Within this environment, some commands are defined:
%   \grule[description]{non-terminal}{productions}
%   \gor (a symbol for separating productions)
%   \gline (a symbol for breaking a line of productions if more space is needed)
%   \setGrammarVertAdjustment{length} (which sets the vertical adjustment for
%                                      the next grammar newline)
\newcommand{\grammarNoteSpace}{\hspace{4mm}}
\newcounter{grammarnote}
\setcounter{grammarnote}{0}
\newcommand{\grammarDefs}{%
    \global\def\grammarVertAdjustment{0mm}%
    \newcommand{\gcomment}[1]{\hfill%
        \ifnum\value{grammarnote}=0
            \stepcounter{grammarnote}
            \grammarNoteSpace \textrm{\textsmaller{\itshape ##1}}
        \fi
    }%
    \newcommand{\grule}[3][]{%
        \setcounter{grammarnote}{0}
        ##2 & \ifnotempty{##3}{::=} & \newcommand{\gcommenttext}{##1} ##3 \hfill \gcomment{##1} \endgrule
    }%
    \newcommand{\gskip}{&&\endgrule}%
    \def\endgrule{\\[\grammarVertAdjustment]}%
    \newcommand{\gor}{\mathrel{\vert}}%
    \newcommand{\gline}{%
        \hfill \gcomment{\gcommenttext} \\[\grammarVertAdjustment] &&
    }%
}
\def\grammarColPad{\quad}%
\newenvironment{grammar}{%
    \begingroup%
    \grammarDefs%
    \begin{math}\begin{array}{@{}r@{\grammarColPad}c@{\grammarColPad}l@{}}%
}{%
    \end{array}\end{math}%
    \endgroup%
}

%%% CoPylot grammars %%%%%%%%%%%%%%%%%%%%%%%%%%%%%%%%%%%%%%%%%%%%%%%%%%%%%%%%%%

%%% Operational semantics

\defgn{olbl}{\ell}
\defgn{oglbl}{\overset{\star}{\ell}}
\defgn{ovalue}{v}
\defgn{ovariable}{x}
\defgn{odirective}{d}
\defgn{ostmt}{s}
\defgn{oprogram}{S}
\defgn{ostack}{T}
\defgn{ostackframe}{t}
\defgn{obinding}{B}
\defgn{oheap}{H}
\defgn{opscope}{\eta}
\defgn{oparent}{P}
\defgn{omem}{m}
\defgn{ogmem}{\overset{\star}{m}}
\defgn{oexpr}{e}
% Function related
\defgn{ofuncarr}{\rightarrow}
\defgn{omagicf}{\mathfrak{F}}
\defgn{omagicm}{\mathfrak{M}}
\defgn{ogenf}{F}
\defgn{ogenm}{M}
\defgn{osplat}{\ast}
% Code stack related
\defgn{omstack}{Y}
\defgn{omcode}{y}
\defgn{olstack}{Z}
\defgn{olcode}{z}

% These are wrong and should not emerge.
% \defgn{oenv}{E}
% \defgn{ofunc}{fun}
% \newcommand{\opscope}{{\color{red}\eta}}
\newcommand{\oenv}{{\color{red}E}}

\newlength\dunder
\settowidth{\dunder}{\_}

% Special
\newcommand{\osnone}{\texttt{None}}
\newcommand{\ostrue}{\text{\textsmaller{{\color{func}\sc True}}}}
\newcommand{\osfalse}{\text{\textsmaller{{\color{func}\sc False}}}}
% Relations
\newcommand{\osTransition}{\mathrel{\longrightarrow^1}}
\newcommand{\osBefore}[1]{\mathrel{\stackrel{{\scriptscriptstyle\boldsymbol{#1}}}{\lexicallyBefore}}}
% Helpers
\newcommand{\osLookup}{\text{\textsmaller{{\color{func}\sc Lookup}}}}
\newcommand{\osWrap}{\text{\textsmaller{{\color{func}\sc Wrap}}}}
\newcommand{\osGetCall}{\text{\textsmaller{{\color{func}\sc GetCall}}}}
% Format
\newcommand{\osLR}[1]{\langle #1 \rangle}
\newcommand{\osI}[2]{#1 &= #2}
\newcommand{\osInStack}[1]{\olstack \ifthenelse {\equal{#1}{}} {\listConcat} {\ifthenelse {\equal{#1}{\omstack'}}{\listConcat #1 \listConcat}{\listConcat [#1] \listConcat}} \omstack}
% Stylize
\newcommand{\okw}[1]{\;\text{\texttt{{\color{kw}#1}}}\;}
\newcommand{\osMC}[1]{\text{\textsmaller{{\color{microcode}\sc #1}}}}
\newcommand{\osFunc}[1]{\text{\textsmaller{{\color{func}\sc #1}}}}
\newcommand{\osType}[1]{\text{\textsmaller{{\color{type}\sc #1}}}}
\newcommand{\osMagic}[1]{\text{{\color{magic}\sc $\mathfrak{#1}$}}}
% -fix
\newcommand{\ostarvalue}[1]{\star \ovariable_{\texttt{#1}}}
\newcommand{\osInit}[1]{#1_{\text{\textsmaller{{\sc Init}}}}}
\newcommand{\osMem}[1]{\omem_{\texttt{#1}}}
\newcommand{\osClown}[1]{\rule{2\dunder}{0.4pt} #1 \rule{2\dunder}{0.4pt}}
% Magic
\newcommand{\osDuoMagic}[6]{
  % #1: magic name, #2: first type, #3: second type list
  % #4: len(list), #5: MI to call, #6: other things to put before the micro-stack
  \relationRule{#1}{%
    \ovalue = \osMagic{#1} \\%
    \osFunc{CheckArgs}(\ovalue, 2) = \ostrue \\
    \ostack = [\osLR{\opscope, \olbl}] \listConcat \ostack' \\%
    \oprogram(\olbl) = \olbl:\oglbl': \ovariable \gteq \oexpr \\%
    #6%
    \omstack' = [\omem_1, \osMC{Get}, \ostarvalue{value}, \osMC{Retrieve}, \osMC{Get}, #2, \osMC{Assert} \;1, \omem_2, \osMC{Get}, \ostarvalue{value}, \osMC{Retrieve}, \\ \osMC{Get}, \ifthenelse {\equal{#3}{}} {} {#3, \osMC{Assert} \;#4, }\osMC{#5}, \ovariable, \osMC{Assign}, \osMC{Advance}]%
  }{%
    \osInStack{\ovalue, \omem_1, \omem_2, \osMC{Call} \;2}, \ostack, \oheap \osTransition%
    \osInStack{\omstack'}, \ostack, \oheap%
  }%
}

% Used in main
\newcommand{\osend}{\text{\textsmaller{{\color{func}\sc End}}}}
\newcommand{\osBind}{\text{\textsmaller{{\color{func}\sc Bind}}}}
\newcommand{\osMakeObj}{\text{\textsmaller{{\color{func}\sc MakeObj}}}}
\newcommand{\osCatch}{\text{\textsmaller{{\color{func}\sc Catch}}}}

%%% Graph semantics

\defgn{ggraph}{G}
\defgn{gedge}{g}
\defgn{gnode}{a}
\defgn{gtime}{\chi}
\defgn{genter}{\llparenthesis}
\defgn{gleave}{\rrparenthesis}
\defgn{gresult}{R}
\defgn{gvset}{V}

\newcommand{\niton}{\not\owns}

% Special
\newcommand{\gsstart}{\text{\textsmaller{{\color{func}\sc Start}}}}
\newcommand{\gsend}{\text{\textsmaller{{\color{func}\sc End}}}}
% Relations
\newcommand{\gsTransition}{\mathrel{\longrightarrow^1}}
\newcommand{\gsBefore}{\ll}
\newcommand{\gsSkip}{\lll}
\newcommand{\gsPrecede}{\ll \kern -4pt \ll}
% Format
\newcommand{\gsLR}[1]{\langle #1 \rangle}
\newcommand{\gsSet}[1]{\{ #1 \}}
%Stylize
\newcommand{\gsFunc}[1]{\text{\textsmaller{{\color{func}\sc #1}}}}

\newcommand{\gsAddEdge}[1]{\ggraph \gsTransition \ggraph \cup \gsSet{#1}}



%%% Abstract language grammar

\defgt[gtcolon]{:}
\defgt[gteq]{=}

\defgt[gtintplus]{int\ensuremath{{}^{\texttt{+}}}}
\defgt[gtintminus]{int\ensuremath{{}^{\texttt{-}}}}
\defgt[gtintzero]{int\ensuremath{{}^{\texttt{0}}}}

\newcommand{\xstmt}[3]{#1\gtcolon#2\gtcolon#3}

%%% Analysis grammar

\defgn{albl}{\hat{\ell}}
\defgn{avalue}{\hat{v}}
\defgn{avariable}{\hat{x}}
\defgn{adirective}{\hat{d}}
\defgn{astmt}{\hat{s}}
\defgn{aprogram}{\hat{S}}

% this will likely be replaced later when we move to a different kind of answer
\defgn{avalues}{\hat{V}}

\defgn{cfgnode}{\hat{o}}
\defgn{cfgedge}{\hat{g}}
\defgn{cfg}{\hat{G}}

\newcommand{\gstart}{\text{\textsmaller{\sc Start}}}
\newcommand{\gend}{\text{\textsmaller{\sc End}}}

\newcommand{\before}{\mathrel{\mathrlap{<}{\ <}}}
\newcommand{\lexicallyBefore}{\blacktriangleleft}

\defgn{astackelement}{\hat{k}}
\defgn{astack}{\hat{K}}

\newcommand{\kcapture}[1]{\text{\textsmaller{\textsc{Capture}}}(#1)}
\newcommand{\kjump}[1]{\text{\textsmaller{\textsc{Jump}}}(#1)}

%%% CoPylot functions and relations %%%%%%%%%%%%%%%%%%%%%%%%%%%%%%%%%%%%%%%%%%%

\newcommand{\isbefore}{\mathrel{\stackrel{{\scriptscriptstyle\boldsymbol{?}}}{\before}}}
\newcommand{\islexicallyBefore}{\mathrel{\stackrel{{\scriptscriptstyle\boldsymbol{?}}}{\lexicallyBefore}}}

\newcommand{\valueLookup}[3][\cfg]{#1(#2,#3)}

\newcommand{\cfgClosureStep}{\mathrel{\longrightarrow^1}}
\newcommand{\cfgClosureSteps}{\mathrel{\longrightarrow^*}}

\usepackage{enumitem}

\begin{document}
    \section{CoPylot}

    \subsection{Grammar}



        \begin{figure}\center
              \begin{grammar}
                \grule[labels]{\oglbl}{\olbl \gor \gend}
                \grule[stack]{\ostack}{
                            [\ostackframe,\ldots]
                }
                \grule[stack frame]{\ostackframe}{\oglbl \times \oprogram}
                \grule[programs]{\oprogram}{
                            [\ostmt,\ldots]
                }
                \grule[directives]{\odirective}{
                            \ovariable \gteq \ovariable
                    \gor    \ovariable \gteq \ovalue
                    \gor    \ovariable \gteq \ovariable(\ovariable)

                }
                \grule[bindings]{\obinding}{\ovariable \mapsto \omem, \ldots}
                \grule[heap]{\oheap}{\omem \mapsto \ovalue, \ldots}
                \grule[values]{\ovalue}{
                            \gtintplus
                    \gor    \gtintminus
                    \gor    \gtintzero
                    \gor    \langle \opscope,\ofunc(x) \mapsto \oprogram \rangle
                }
                \grule[pointers to scopes]{\opscope}{}
                \grule[environments]{\oenv}{\opscope \mapsto \obinding}
                \grule[parental maps]{\oparent}{\opscope \mapsto \opscope}

              \end{grammar}

              \begin{mathpar}
              \relationRule{Literal Assignment}{
                  \oprogram(\olbl) \gteq \olbl:\olbl':\ovariable=\ovalue \\
                  \oheap' \gteq \oheap[\omem \mapsto \ovalue] \\
                  \omem \notin \oheap \\
                  \obinding \gteq \oenv(\opscope) \\
                  \obinding' \gteq \obinding[\ovariable \mapsto \omem] \\
                  \oenv' \gteq \oenv[\opscope \mapsto \obinding'] \\
                  \olbl \osBefore \olbl''
              }{
                  [\langle \olbl,\oprogram \rangle] \listConcat \ostack,\oheap,\oenv,\oparent,\opscope \osTransition
                  [\langle \olbl'',\oprogram \rangle] \listConcat \ostack,\oheap',\oenv',\oparent,\opscope
              }

              \relationRule{Variable Assignment}{
                  \oprogram(\olbl) \gteq \olbl:\olbl':\ovariable_1=\ovalue_2 \\
                  \omem \gteq \osLookup (\opscope,\oparent,\oenv,\ovariable_1)\\
                  \obinding \gteq \oenv(\opscope) \\
                  \obinding' \gteq \obinding[\ovariable_1 \mapsto \omem] \\
                  \oenv' \gteq \oenv[\opscope \mapsto \obinding'] \\
                  \olbl \osBefore \olbl''
              }{
                  [\langle \olbl,\oprogram \rangle] \listConcat \ostack,\oheap,\oenv,\oparent,\opscope \osTransition
                  [\langle \olbl'',\oprogram \rangle] \listConcat \ostack,\oheap,\oenv',\oparent,\opscope
              }

              \relationRule{Function Call}{
                  \oprogram(\olbl) \gteq \olbl:\olbl':\ovariable_1=\ovariable_2(\ovariable_3) \\
                  \omem \gteq \ocLookup (\opscope,\oparent,\oenv,\ovariable_2)\\
                  \oheap[\omem] \gteq \langle \opscope',\ofunc(\ovariable_4) \mapsto \oprogram' \rangle \\
                  \opscope'' \notin \oenv \\
                  \omem' \gteq \ocLookup (\opscope,\oparent,\oenv,\ovariable_3) \\
                  \obinding \gteq \{\ovariable_4 \mapsto \omem'\} \\
                  \oenv' \gteq \oenv[\opscope'' \mapsto \obinding] \\
                  \oparent' \gteq \oparent \cup \{\opscope'' \mapsto \opscope'\} \\
                  \oprogram' \gteq [\olbl'':\olbl''':\odirective]
              }{
                  [\langle \olbl,\oprogram \rangle] \listConcat \ostack,\oheap,\oenv,\oparent,\opscope \osTransition
                  [\langle \olbl'',\oprogram' \rangle,\langle l,S \rangle] \listConcat \ostack,\oheap',\oenv',\oparent',\opscope''
              }
            \end{mathpar}
            \caption{Operation Semantics}
            \label{fig_languageOS}
        \end{figure}

    \begin{figure}\center
        \begin{grammar}
            \grule[abstract programs]{\aprogram}{
                        [\astmt,\ldots]
            }
            \grule[abstract statements]{\astmt}{
                        \xstmt{\albl}{\albl}{\adirective}
            }
            \grule[abstract directives]{\adirective}{
                        \avariable \gteq \avalue
                \gor    \avariable \gteq \avariable
            }
            \grule[abstract values]{\avalue}{
                        \gtintplus
                \gor    \gtintminus
                \gor    \gtintzero
            }
            \grule[abstract variables]{\avariable}{}
            \grule[abstract labels]{\albl}{}

        \end{grammar}
        \caption{Normalized Python Language Grammar}
        \label{fig_languageGrammar}
    \end{figure}

    The grammar of the language to be analyzed appears in Figure~\ref{fig_languageGrammar}.

    We assume throughout the rest of this document that a fixed program $\aprogram$ is under analysis.  \znote{TODO: describe here the idea of a bijection between labels and statements in this fixed program.}

    \subsection{Control Flow}

    The grammar of control flow graphs appears in Figure~\ref{fig_cfgGrammar}.  \znote{Discuss construction of initial graph.}

    \begin{figure}\center
        \begin{grammar}
            \grule[control flow graphs]{\cfg}{
                        \{\cfgedge, \ldots\}
            }
            \grule[control flow graph edge]{\cfgedge}{
                        \cfgnode \lexicallyBefore \cfgnode
                \gor    \cfgnode \before \cfgnode
            }
            \grule[control flow graph nodes]{\cfgnode}{
                        \gstart
                \gor    \gend
                \gor    \astmt
            }
        \end{grammar}
        \caption{Control Flow Graph Grammar}
        \label{fig_cfgGrammar}
    \end{figure}



    We write $\cfgnode \islexicallyBefore \cfgnode'$ to denote $(\cfgnode \lexicallyBefore \cfgnode' \in \cfg$ when $\cfg$ is understood from context.  Likewise, we write $\cfgnode \isbefore \cfgnode'$ to denote $(\cfgnode \before \cfgnode' \in \cfg$ when $\cfg$ is understood from context.

    We define a relation $\cfgClosureStep$ to perform control flow graph closure.

    \begin{definition}
        Let $\cfg \cfgClosureStep \cfg'$ be the least relation satisfying the rules appearing in Figure~\ref{fig_cfgClosure}.  Throughout these rules, the predicates $\isbefore$ and $\islexicallyBefore$ refer to graph $\cfg$.
    \end{definition}

    \begin{figure}
        \begin{mathpar}
            \relationRule{Lexical Start}{
                \gstart \islexicallyBefore \cfgnode
            }{
                \cfg \cfgClosureStep \cfg \cup \{\gstart \before \cfgnode\}
            }

            \relationRule{Literal Assignment}{
                \cfgnode_1 = (\avariable \gteq \avalue) \\
                \cfgnode_1 \islexicallyBefore \cfgnode_2
            }{
                \cfg \cfgClosureStep \cfg \cup \{\cfgnode_1 \before \cfgnode_2\}
            }

            \relationRule{Variable Accessible}{
                \cfgnode_1 = (\avariable \gteq \avariable') \\
                \cfgnode_1 \islexicallyBefore \cfgnode_2 \\
                \avalue \in \valueLookup{\cfgnode_1}{[\avariable']} \\
                \avalue \neq {\text{\textsmaller{\sc Undefined}}}
            }{
                \cfg \cfgClosureStep \cfg \cup \{\cfgnode_1 \before \cfgnode_2\}
            }
        \end{mathpar}
        \caption{Control Flow Graph Closure}
        \label{fig_cfgClosure}
    \end{figure}

    \subsection{Value Lookup}

    The value lookup function uses the additional grammar in Figure~\ref{fig_valueLookupGrammar}.

    \begin{figure}
        \begin{grammar}
            \grule[lookup stacks]{\astack}{
                        [\astackelement, \ldots]
            }
            \grule[lookup stack elements]{\astackelement}{
                        \avariable
                \gor    \avalue % subject to change to e.g. heap fragments
                \gor    \kcapture{\mathbb{N}}
                \gor    \kjump{\cfgnode}
            }
        \end{grammar}
        \caption{Value Lookup Grammar}
        \label{fig_valueLookupGrammar}
    \end{figure}

    \begingroup
        \newenvironment{enumerateClauses}%
            {\begin{enumerate}[label=(\alph*),ref=\arabic{enumi}\alph*]}%
            {\end{enumerate}}
        \newenvironment{enumerateSubclauses}%
            {\begin{enumerate}[label=\roman*.,ref=\arabic{enumi}\alph{enumii}(\roman*)]}%
            {\end{enumerate}}
        \newcommand{\clauseSectionTitle}[1]{\textbf{#1}}
        \newcommand{\clauseSubsectionTitle}[1]{\underline{\smash{#1}}}
        \newcommand{\rulename}[1]{%
            \begingroup%
                \setlength{\fboxsep}{1.5pt}%
                \fcolorbox{black}{gray!15!white}{%
                    \textsc{\textsmaller{#1}}%
                }%
                \vspace*{1pt}%
            \endgroup%
            \\%
        }
        \begin{definition}
            Given a control-flow graph $\cfg$, let $\valueLookup[\cfg]{\cfgnode_0}{\astack}$ be the function returning the least set $\avalues$ which satisfies the following conditions:

            \begin{enumerate}
                \item \clauseSectionTitle{Value Manipulation}
                \begin{enumerateClauses}
                    \item \rulename{Result}
                        If
                            $\astack = [\avalue]$,
                        then
                            $\avalue \in \avalues$.
                \end{enumerateClauses}

                \item \clauseSectionTitle{Variable Lookup}
                \begin{enumerateClauses}
                    \item \rulename{Value Discovery}
                        If
                            $\cfgnode_1 \isbefore \cfgnode_0$,
                            $\cfgnode_1 =
                                \xstmt{\albl_1}{\albl_2}{\avariable \gteq \avalue}$, and
                            $\astack = [\avariable] \listConcat \astack'$,
                        then
                            $\valueLookup{\cfgnode_1}{[\avalue] \listConcat \astack'} \subseteq \avalues$.

                    \item \rulename{Value Skip}
                        If
                            $\cfgnode_1 \isbefore \cfgnode_0$,
                            $\cfgnode_1 =
                                \xstmt{\albl_1}{\albl_2}{\avariable' \gteq \avalue}$,
                            $\astack = [\avariable] \listConcat \astack'$, and
                            $\avariable \neq \avariable'$,
                        then
                            $\valueLookup{\cfgnode_1}{\astack} \subseteq \avalues$.
                    \item \rulename{Value Aliasing}
                        If
                            $\cfgnode_1 \isbefore \cfgnode_0$,
                            $\cfgnode_1 =
                                \xstmt{\albl_1}{\albl_2}{\avariable \gteq \avariable'}$, and
                            $\astack = [\avariable] \listConcat \astack'$,
                        then
                            $\valueLookup{\cfgnode_1}{[\avariable'] \listConcat \astack} \subseteq \avalues$.
                \end{enumerateClauses}


            \end{enumerate}
        \end{definition}
        \endgroup
\end{document}
